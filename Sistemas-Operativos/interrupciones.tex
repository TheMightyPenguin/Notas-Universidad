\documentclass{article}

\usepackage[spanish]{babel}
\usepackage[utf8]{inputenc}
\usepackage[T1]{fontenc}
\usepackage{graphicx}
\usepackage{wrapfig}
\usepackage{hyperref}
\usepackage{courier}
\usepackage{listings}
\usepackage{xcolor}
 % % % % % % % %

\usepackage{anysize}
\marginsize{2.54cm}{2.54cm}{2.54cm}{2.54cm}

\usepackage{setspace}
\onehalfspacing

%En caso de que LaTeX separe las palabras con - de manera incorrecta, usar
%\hyphenation{deci-sión,e-xa-men, otras palabras....}

% %PORTADA
\begin{document}

\centerline{Universidad de Carabobo}
\centerline{Facultad de Ciencia y Tecnologia}
\centerline{Sistemas Operativos}

\begin{centering}
\hrule 	\vspace{0.4cm}
	{ \Huge \bfseries Interrupciones \\[0.4cm] }
\hrule \vfill
\end{centering}

\noindent
\centerline{Victor Tortolero\newline

\vspace{8cm}

\centerline{\today}

\newpage
% %Fin Portada


\flushleft
\setlength{\parindent}{20pt}

%%CUERPO PRINCIPAl%%%%%%%%%%%%%%%%%%%%%%%%%%%%%%%%%%%%%%%%%%%
\begin{centering} \section{Las Interrupciones} \end{centering}
	Una interrupción es una señal que recibe el procesador, ante la cual debe detener cualquier proceso que este realizando y darle prioridad al proceso o tarea que debe realizar según la interrupción que haya ocurrido. Las interrupciones también puede decirse que son un evento, que es accionado por una señal de hardware o software. 
	
	Las interrupciones de hardware son usadas para indicar al sistema operativo que cierto dispositivo requiere atención. Y fueron hechas con la finalidad de evitar código innecesario que solo esperaba por una entrada o señal de algún hardware.
	
	Cuando hablamos de las interrupciones por software, podemos hablar de las excepciones, que se usan para que el programa no reaccione de manera inesperada y sepa manejar los distintos errores que podrían causarse en tiempo de ejecución. Por ejemplo, en el caso de que el usuario ingrese una cadena y el programa espere un entero, en ese caso el programa no terminaría con un error, sino que haría algo al respecto, como notificarle al usuario y seguiría con la ejecución normal.Las interrupciones son comúnmente usadas para programas multitarea.
\newpage
	
	
\begin{centering} \section{Programacion Concurrente} \end{centering}
	Se habla de concurrencia cuando ocurren varios sucesos de manera contemporánea.
	En base a esto, la concurrencia en computación esta asociada a la ejecución de varios procesos que coexisten temporalmente.
	
	Para definir correctamente la programación concurrente, es necesario diferenciar programa de proceso.
	Un programa es un conjunto de sentencias o instrucciones que se ejecutan secuencialmente, y un proceso es básicamente un programa en ejecución.
	
	La concurrencia aparece cuando dos o mas procesos son contemporáneos. Un caso particular es el paralelismo o programación paralela. Los procesos pueden competir o colaborar entre si por los recursos del sistema. Por tanto, existen tareas de colaboración y sincronización
	
	La programación concurrente se encarga del estudio de las nociones de ejecución concurrente, asi como sus problemas de comunicación y sincronización.

\vspace{1cm}
\begin{centering} \section{Ejecución concurrente de programas y el temporizador} \end{centering}
	En programación concurrente un temporizador es un objeto que puede notificar a un proceso si ha transcurrido un cierto intervalo de tiempo o se ha alcanzado una hora determinada, los temporizadores permiten crear periodos temporales que una vez finalizados pueden generar señales que se envíen a otros procesos o al sistema operativo, informando sobre la finalización del temporizador. De igual forma también se pueden utilizar como contadores de cuenta atrás. cada temporizador esta asociado a un reloj.
%%FIN CUERPO PRINCIPAl%%%%%%%%%%%%%%%%%%%%%%%%%%%%%%%%%%
	
\end{document}
