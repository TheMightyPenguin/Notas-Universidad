\documentclass{beamer}

%--% Paquetes %----------------------------------%
\usepackage[spanish]{babel}
\usepackage[utf8]{inputenc}
\usepackage[T1]{fontenc}
\usepackage{graphicx}
\usepackage{hyperref}
\usepackage{courier}
\usepackage{listings}
\usepackage{xcolor}
\usepackage{blindtext}
\usepackage{scrextend}
\usepackage[document]{ragged2e}
\usepackage{multicol}
\usepackage{pgfgantt}
\usepackage{minted}
\usepackage{tikz}
\usepackage{longtable}
\usepackage{algorithm}
\usepackage[noend]{algpseudocode}
\usepackage{amsmath}
\usepackage{wrapfig,lipsum,booktabs}
%------------------------------------------------%

\usetikzlibrary{positioning,fit,calc}

\makeatletter
\newcommand*{\MoveFitHeight}[1]{%
	\pgfmathsetlengthmacro\fit@inner@sep{%
		\pgfkeysvalueof{/pgf/inner xsep}%
	}%
	\pgfmathsetlengthmacro\fit@text@height{%
		\tikz@text@height
	}%
	\kern-\fit@inner@sep\relax
	\raisebox{\fit@text@height}[0pt][0pt]{#1}%
}
\makeatother

\newcommand{\algTitle}{\textbf{Algoritmo}: }

%--% Personal Info %-----------------------------%
\title{Algoritmos de Planificación del Procesador}
\author{Victor Tortolero}
\institute{
	Universidad de Carabobo \\
	Facultad de Ciencia y Tecnología \\
	Sistemas Operativos
}
\date{\today}
%------------------------------------------------%

\begin{document}

\begin{frame}
	\titlepage
\end{frame}

% FCSFS
\begin{frame}
\frametitle{\algTitle First Come, First Serve (FCFS)}
\begin{itemize}
	\item \textbf{Ventajas}
	\begin{itemize}
		\item Es fácil de entender e implementar.
		\item Aunque normalmente es justo en como dedica tiempo de CPU a los procesos,
		los procesos largos hacen esperar a los cortos.
	\end{itemize}
	
	\item \textbf{Desventajas}
	\begin{itemize}
		\item El tiempo de espera es alto por lo que carece de rendimiento.
		\item No apropiativo.
	\end{itemize}
\end{itemize}

\begin{figure}[h]
	\centering
	\begin{minipage}{0.45\textwidth}
		\centering
		\begin{tikzpicture}[box/.style={draw,minimum width=2cm, minimum height=1cm,align=center}, node distance=0cm and 0cm]
		\node[box, minimum width=3cm] (N1) {$P_{0}$};
		\node[box, minimum width=0.4cm, right=of N1] (N2) {$P_{1}$};
		\node[box, minimum width=0.8cm, right=of N2] (N3) {$P_{2}$};
		\node[, dashed, inner sep=2pt, fit={(N1) (N2)}, align=left,] (fit) {\MoveFitHeight{0}};
		\node[, dashed, inner sep=2pt, fit={(N2) (N3)}, align=left,] (fit) {\MoveFitHeight{20}};
		\node[, dashed, inner sep=2pt, fit={(N3) (N3)}, align=left,] (fit) {\MoveFitHeight{26}};
		\node[, dashed, inner sep=2pt, fit={(N3) (N3)}, align=right,] (fit) {\MoveFitHeight{35}};
		\end{tikzpicture}	
		\caption{FCFS, Ejemplo 1}
		\label{gantt:FCFS1}
	\end{minipage}\hfill
	\begin{minipage}{0.45\textwidth}
		\centering
		\begin{tikzpicture}[box/.style={draw,minimum width=2cm, minimum height=1cm,align=center}, node distance=0cm and 0cm]
		\node[box, minimum width=0.4cm] (N1) {$P_{1}$};
		\node[box, minimum width=0.8cm, right=of N1] (N2) {$P_{2}$};
		\node[box, minimum width=3cm, right=of N2] (N3) {$P_{0}$};
		\node[, dashed, inner sep=2pt, fit={(N1) (N2)}, align=left,] (fit) {\MoveFitHeight{0}};
		\node[, dashed, inner sep=2pt, fit={(N2) (N3)}, align=left,] (fit) {\MoveFitHeight{6}};
		\node[, dashed, inner sep=2pt, fit={(N3) (N3)}, align=left,] (fit) {\MoveFitHeight{15}};
		\node[, dashed, inner sep=2pt, fit={(N3) (N3)}, align=right,] (fit) {\MoveFitHeight{35}};
		\end{tikzpicture}
		\caption{FCFS, Ejemplo 2}
		\label{gantt:FCFS2}
	\end{minipage}
\end{figure}
\end{frame}

% Round Robin
\begin{frame}
\frametitle{\algTitle Round Robin (RR)}
\begin{itemize}
	\item \textbf{Ventajas}
	\begin{itemize}
		\item Es justo con todos los procesos.
		\item Es apropiativo.
	\end{itemize}
	
	\item \textbf{Desventajas}
	\begin{itemize}
		\item Si el valor del quantum es mayor que el tiempo requerido 
		por el proceso mas largo, se convierte en FCFS.
		\item Si el valor del quantum es muy pequeño se producen muchos
		cambios de contexto lo que es ineficiente.
	\end{itemize}
\end{itemize}
\begin{figure}[h]
	\centering
	\begin{minipage}{0.45\textwidth}
		\centering
		\begin{tikzpicture}[box/.style={draw,minimum width=2cm, minimum height=1cm,align=center}, node distance=0cm and 0cm]
		\node[box, minimum width=1.2cm] (N1) {$P_{0}$};
		\node[box, minimum width=0.9cm, right=of N1] (N2) {$P_{1}$};
		\node[box, minimum width=1.2cm, right=of N2] (N3) {$P_{2}$};
		\node[box, minimum width=0.6cm, right=of N3] (N4) {$P_{1}$};
		\node[box, minimum width=1.2cm, right=of N4] (N5) {$P_{2}$};
		\node[dashed, inner sep=2pt, fit={(N1) (N2)}, align=left,] (fit) {\MoveFitHeight{0}};
		\node[dashed, inner sep=2pt, fit={(N2) (N3)}, align=left,] (fit) {\MoveFitHeight{4}};
		\node[dashed, inner sep=2pt, fit={(N3) (N4)}, align=left,] (fit) {\MoveFitHeight{7}};
		\node[dashed, inner sep=2pt, fit={(N4) (N5)}, align=left,] (fit) {\MoveFitHeight{11}};
		\node[dashed, inner sep=2pt, fit={(N5) (N5)}, align=left,] (fit) {\MoveFitHeight{12}};
		\node[dashed, inner sep=2pt, fit={(N5) (N5)}, align=right,] (fit) {\MoveFitHeight{16}};
		\end{tikzpicture}
		\caption{RR, Q = 4}
		\label{gantt:RR1}
	\end{minipage}\hfill
	\begin{minipage}{0.45\textwidth}
		\centering
		\begin{tikzpicture}[box/.style={draw,minimum width=2cm, minimum height=1cm,align=center}, node distance=0cm and 0cm]
		\node[box, minimum width=1.2cm] (N1) {$P_{0}$};
		\node[box, minimum width=0.9cm, right=of N1] (N2) {$P_{1}$};
		\node[box, minimum width=1.2cm, right=of N2] (N3) {$P_{2}$};
		\node[dashed, inner sep=2pt, fit={(N1) (N2)}, align=left,] (fit) {\MoveFitHeight{0}};
		\node[dashed, inner sep=2pt, fit={(N2) (N3)}, align=left,] (fit) {\MoveFitHeight{5}};
		\node[dashed, inner sep=2pt, fit={(N3) (N3)}, align=left,] (fit) {\MoveFitHeight{8}};
		\node[dashed, inner sep=2pt, fit={(N3) (N3)}, align=right,] (fit) {\MoveFitHeight{16}};
		\end{tikzpicture}
		\caption{RR, Q = 8}
		\label{gantt:RR2}
	\end{minipage}\hfill
\end{figure}
\end{frame}
	

% Prioridades
\begin{frame}
\frametitle{\algTitle Prioridades}
\begin{itemize}
	\item \textbf{Ventajas}
	\begin{itemize}
		\item Puede ser apropiativo o no apropiativo.
	\end{itemize}
	
	\item \textbf{Desventajas}
	\begin{itemize}
		\item Si no se usa envejecimiento, un proceso con muy baja prioridad puede llegar
		a no ejecutarse nunca.
	\end{itemize}
\end{itemize}

\begin{figure}[h]
	\centering
	\begin{minipage}{0.45\textwidth}
		\centering
		\begin{tikzpicture}[box/.style={draw,minimum width=2cm, minimum height=1cm,align=center}, node distance=0cm and 0cm]
		\node[box, minimum width=0.6cm] (N1) {$P_{0}$};
		\node[box, minimum width=0.9cm, right=of N1] (N2) {$P_{1}$};
		\node[box, minimum width=1.6cm, right=of N2] (N3) {$P_{2}$};
		\node[box, minimum width=1.1cm, right=of N3] (N4) {$P_{0}$};
		\node[, dashed, inner sep=2pt, fit={(N1) (N2)}, align=left,] (fit) {\MoveFitHeight{0}};
		\node[, dashed, inner sep=2pt, fit={(N2) (N3)}, align=left,] (fit) {\MoveFitHeight{1}};
		\node[, dashed, inner sep=2pt, fit={(N3) (N4)}, align=left,] (fit) {\MoveFitHeight{4}};
		\node[, dashed, inner sep=4pt, fit={(N4) (N4)}, align=left,] (fit) {\MoveFitHeight{12}};
		\node[, dashed, inner sep=2pt, fit={(N4) (N4)}, align=right,] (fit) {\MoveFitHeight{16}};
		\end{tikzpicture}
		\caption{Prioridades, Sin envejecimiento}
		\label{gantt:Prioridades1}
	\end{minipage}\hfill
	\begin{minipage}{0.45\textwidth}
		\centering
		\begin{tikzpicture}[box/.style={draw,minimum width=2cm, minimum height=1cm,align=center}, node distance=0cm and 0cm]
		\node[box, minimum width=0.6cm] (N1) {$P_{0}$};
		\node[box, minimum width=0.9cm, right=of N1] (N2) {$P_{1}$};
		\node[box, minimum width=1.1cm, right=of N2] (N3) {$P_{0}$};
		\node[box, minimum width=1.6cm, right=of N3] (N4) {$P_{2}$};
		\node[, dashed, inner sep=2pt, fit={(N1) (N2)}, align=left,] (fit) {\MoveFitHeight{0}};
		\node[, dashed, inner sep=2pt, fit={(N2) (N3)}, align=left,] (fit) {\MoveFitHeight{1}};
		\node[, dashed, inner sep=2pt, fit={(N3) (N4)}, align=left,] (fit) {\MoveFitHeight{4}};
		\node[, dashed, inner sep=4pt, fit={(N4) (N4)}, align=left,] (fit) {\MoveFitHeight{8}};
		\node[, dashed, inner sep=2pt, fit={(N4) (N4)}, align=right,] (fit) {\MoveFitHeight{16}};
		\end{tikzpicture}
		\caption{Prioridades, Envejecimiento T=2}
		\label{gantt:Prioridades2}
	\end{minipage}\hfill
\end{figure}
\end{frame}



\end{document}