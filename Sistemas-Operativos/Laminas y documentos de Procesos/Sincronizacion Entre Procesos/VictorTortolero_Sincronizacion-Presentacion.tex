\documentclass{beamer}

%--% Paquetes %----------------------------------%
\usepackage[spanish]{babel}
\usepackage[utf8]{inputenc}
\usepackage[T1]{fontenc}
\usepackage{graphicx}
\usepackage{hyperref}
\usepackage{courier}
\usepackage{listings}
\usepackage{xcolor}
\usepackage{blindtext}
\usepackage{scrextend}
\usepackage[document]{ragged2e}
\usepackage{multicol}
\usepackage{pgfgantt}
\usepackage{minted}
\usepackage{tikz}
\usepackage{longtable}
\usepackage{algorithm}
\usepackage[noend]{algpseudocode}
\usepackage{amsmath}
\usepackage{wrapfig,lipsum,booktabs}
%------------------------------------------------%

% \usemintedstyle{emacs}

\usetikzlibrary{positioning,fit,calc}

\makeatletter
\newcommand*{\MoveFitHeight}[1]{%
	\pgfmathsetlengthmacro\fit@inner@sep{%
		\pgfkeysvalueof{/pgf/inner xsep}%
	}%
	\pgfmathsetlengthmacro\fit@text@height{%
		\tikz@text@height
	}%
	\kern-\fit@inner@sep\relax
	\raisebox{\fit@text@height}[0pt][0pt]{#1}%
}
\makeatother

\newcommand{\algTitle}{\textbf{Algoritmo}: }
\newcommand{\problemasTitle}{\textbf{Problemas Clasicos}: }

\newcommand{\bigO}[1]{$O({#1})$}

\usetheme{Berlin}
%\usecolortheme{beaver}

%--% Personal Info %-----------------------------%
\title{Algoritmos de Planificación del Procesador}
\author{Victor Tortolero, 24.569.609}
\institute{
	Sistemas Operativos, FACYT
}
\date{\today}
%------------------------------------------------%


\begin{document}
\setbeamertemplate{caption}{\raggedright\insertcaption\par}

\begin{frame}
	\titlepage

\end{frame}
\author{ }


%--% Primera version del algoritmo de Dekker %------------------------------------%
\begin{frame}
\frametitle{Primera version del algoritmo de Dekker}

\begin{itemize}
	\item Primer algoritmo en resolver la exclusión mutua.
	\item Aplica la exclusión mutua de manera correcta, y la garantiza.
	\item Usa variables para controlar que hilo se ejecutara.
	\item Revisa constantemente si la sección critica esta disponible (spinlock, busy waiting),
	lo que malgasta el tiempo del procesador.
	\item Los procesos lentos atrasan a los rápidos.
\end{itemize}
\end{frame}

\begin{frame}
	\frametitle{Primera version del algoritmo de Dekker}
	\inputminted[
		frame=lines,
		framesep=2mm,
		baselinestretch=1.1,
		fontsize=\tiny,
		tabsize=4,
		obeytabs,
		linenos
	]{c}{codigos/dekker1.c}
\end{frame}
%--------------------------------------------------------------------------------%

%--% Segunda version del algoritmo de Dekker %-----------------------------------%
\begin{frame}
	\frametitle{Segunda version del algoritmo de Dekker}
	
	\begin{itemize}		
		\item No garantiza la exclusión mutua.
		\item Ambos procesos pueden entrar al mismo tiempo en su sección critica
	\end{itemize}
\end{frame}

\begin{frame}
	\frametitle{Segunda version del algoritmo de Dekker}
	\inputminted[
	frame=lines,
	framesep=2mm,
	baselinestretch=0.95,
	fontsize=\tiny,
	tabsize=4,
	obeytabs,
	linenos
	]{c}{codigos/dekker2.c}
\end{frame}
%--------------------------------------------------------------------------------%

%--% Tercera version del algoritmo de Dekker %-----------------------------------%
\begin{frame}
	\frametitle{Tercera version del algoritmo de Dekker}
	
	\begin{itemize}
		\item Garantiza la exclusión mutua.
		\item Es posible que ocurra un deadlock.
		
	\end{itemize}	
\end{frame}

\begin{frame}
	\frametitle{Tercera version del algoritmo de Dekker}
	\inputminted[
	frame=lines,
	framesep=2mm,
	baselinestretch=0.95,
	fontsize=\tiny,
	tabsize=4,
	obeytabs,
	linenos
	]{c}{codigos/dekker3.c}
\end{frame}
%--------------------------------------------------------------------------------%

%--% Cuarta version del algoritmo de Dekker %-----------------------------------%
\begin{frame}
	\frametitle{Cuarta version del algoritmo de Dekker}
	
	\begin{itemize}
		\item Es posible posponer un proceso de manera indefinida.
		\item Apaga las banderas por cortos periodos de tiempo para tomar control.
	\end{itemize}
\end{frame}

\begin{frame}
	\frametitle{Cuarta version del algoritmo de Dekker}
	\inputminted[
	frame=lines,
	framesep=2mm,
	baselinestretch=0.85,
	fontsize=\tiny,
	tabsize=4,
	obeytabs,
	linenos
	]{c}{codigos/dekker4.c}
\end{frame}
%--------------------------------------------------------------------------------%

%--% Quinta version del algoritmo de Dekker %-----------------------------------%
\begin{frame}
	\frametitle{Quinta version del algoritmo de Dekker}
	
	\begin{itemize}
		\item Garantiza la exclusión mutua.
		\item Marca procesos como preferidos para determinar el uso de las secciones criticas.
		\item El estatus de ``Preferido'' se turna entre los procesos.
		\item Evita deadlock's, y el posponer un proceso de manera indefinida.
	\end{itemize}	
\end{frame}

\begin{frame}[allowframebreaks]
	\frametitle{Quinta version del algoritmo de Dekker}
	\inputminted[
	frame=lines,
	framesep=2mm,
	baselinestretch=1.1,
	fontsize=\tiny,
	tabsize=4,
	obeytabs,
	linenos
	]{c}{codigos/dekker5.c}
\end{frame}
%--------------------------------------------------------------------------------%

%--% Peterson %-----------------------------------%
\begin{frame}
	\frametitle{Algoritmo de Peterson}
	
	\begin{itemize}
		\item Garantiza la exclusión mutua.
		\item Cada proceso tendra su turno.
		\item Requiere que los 2 procesos conpartan 2 variables:
		\textbf{int} turn; \textbf{boolean} flag[2];
	\end{itemize}	
\end{frame}

\begin{frame}[allowframebreaks]
	\frametitle{Quinta version del algoritmo de Dekker}
	\inputminted[
	frame=lines,
	framesep=2mm,
	baselinestretch=1.1,
	fontsize=\tiny,
	tabsize=4,
	obeytabs,
	linenos
	]{c}{codigos/peterson.c}
\end{frame}
%--------------------------------------------------------------------------------%

%--% Test and Set %-----------------------------------%
\begin{frame}
	\frametitle{Test and Set}
	
	\begin{itemize}
		\item Operación Atómica.
		\item Retorna el valor del lock, y lo cambia a verdadero.
		\item Si el valor retornado es \textbf{falso}, obtenemos el lock. Si es \textbf{verdadero}, 
		esta ocupado por otro proceso.
	\end{itemize}	
	
	\inputminted[
	frame=lines,
	framesep=2mm,
	baselinestretch=1.0,
	fontsize=\scriptsize,
	tabsize=4,
	obeytabs,
	linenos
	]{c}{codigos/test_and_set.c}
\end{frame}
%--------------------------------------------------------------------------------%

%--% Compare and Swap %-----------------------------------%
\begin{frame}
	\frametitle{Compare and Swap}
	
	\begin{itemize}
		\item Operación Atómica.
		\item Retorna el valor original de value.
		\item Cambia el valor de la variable value si es igual al valor de expected.
	\end{itemize}	
	
	\inputminted[
	frame=lines,
	framesep=2mm,
	baselinestretch=1.0,
	fontsize=\scriptsize,
	tabsize=4,
	obeytabs,
	linenos
	]{c}{codigos/compare_and_swap.c}
\end{frame}
%--------------------------------------------------------------------------------%

%--% Problemas Clasicos %------------------------------------------%
%--% Productor y Consumidor %-----------------------------------%
\begin{frame}
	\frametitle{\problemasTitle Productor y Consumidor}
	
	\begin{itemize}
		{\scriptsize
			\item Los productores insertan elementos en el buffer. Los consumidores los extraen.
			\item No se pueden insertar elementos en el buffer si esta lleno. No se pueden extraer si esta vació.
			\item \textbf{Semaphore}: (binario)mutex = 1, (contador)empty = n, (contador)full = 0;
		}
	\end{itemize}
	
	\begin{figure}[h]
		\begin{minipage}{0.45\textwidth}
			\centering
			\caption{Proceso Productor}
			\inputminted[
			frame=lines,
			framesep=2mm,
			baselinestretch=0.8,
			fontsize=\tiny,
			tabsize=4,
			obeytabs,
			]{c}{codigos/productor.c}
			\label{algoritmos:productor}
		\end{minipage}\hfill
		\begin{minipage}{0.45\textwidth}
			\centering
			\caption{Proceso Consumidor}
			\inputminted[
			frame=lines,
			framesep=2mm,
			baselinestretch=0.9,
			fontsize=\tiny,
			tabsize=4,
			obeytabs,
			]{c}{codigos/consumidor.c}
			\label{algoritmos:consumidor}
		\end{minipage}
	\end{figure}
\end{frame}
%--------------------------------------------------------------------------------%

%--% Lectores y Escritores %-----------------------------------%
\begin{frame}
	\frametitle{\problemasTitle Lectores y Escritores}
	
	\begin{itemize}
		{\scriptsize
		\item Una base de datos que debe ser compartida por \textbf{lectores} y \textbf{escritores}.
		\item Si dos lectores acceden de manera simultanea, no hay problemas.
		\item Si un escritor y otro proceso (lector o escritor), acceden simultáneamente, puede traer problemas.
		}
	\end{itemize}
	\begin{figure}[h]
		\begin{minipage}{0.45\textwidth}
			\centering
			\caption{Proceso Lector}
			\inputminted[
			frame=lines,
			framesep=2mm,
			baselinestretch=0.8,
			fontsize=\tiny,
			tabsize=4,
			obeytabs,
			]{c}{codigos/lector.c}
			\label{algoritmos:lector}
		\end{minipage}\hfill
		\begin{minipage}{0.45\textwidth}
			\centering
			
			{\tiny 
				\textbf{Semaphore}: (binario)rw\_mutex = 1, (binario)mutex = 1; \textbf{Int} read\_count = 0;
			}
			
			\caption{Proceso Escritor}
			\inputminted[
			frame=lines,
			framesep=2mm,
			baselinestretch=0.9,
			fontsize=\tiny,
			tabsize=4,
			obeytabs,
			]{c}{codigos/escritor.c}
			\label{algoritmos:escritor}
		\end{minipage}
	\end{figure}
\end{frame}
%--------------------------------------------------------------------------------%

%--% Barbero Dormilon %-----------------------------------%
\begin{frame}
	\frametitle{\problemasTitle Barbero Dormilon}
	
	\begin{itemize}
		{\scriptsize
		\item El barbero duerme si no hay clientes.
		\item Si no hay sillas disponibles, el cliente se va. En caso contrario se sienta en una.
		\item Si el cliente llega y el barbero esta dormido, lo despierta.
		}
	\end{itemize}
	\begin{figure}[h]
		\begin{minipage}{0.45\textwidth}
			\centering
			\caption{Proceso Cliente}
			\inputminted[
			frame=lines,
			framesep=2mm,
			baselinestretch=0.9,
			fontsize=\tiny,
			tabsize=4,
			obeytabs,
			]{c}{codigos/cliente.c}
			\label{algoritmos:cliente}
		\end{minipage} \hfill
		\begin{minipage}{0.45\textwidth}
			\centering
			
			{\tiny
			\textbf{Int:} sillas\_libres = N; \textbf{Semaphore:} sillon = 1, barbero = 0, fin = 0, mutex = 1, turn = 1;
			}
			
			\caption{Proceso Barbero}
			\inputminted[
			frame=lines,
			framesep=2mm,
			baselinestretch=0.8,
			fontsize=\tiny,
			tabsize=4,
			obeytabs,
			]{c}{codigos/barbero.c}
			\label{algoritmos:barbero}
		\end{minipage}
	\end{figure}
\end{frame}
%--------------------------------------------------------------------------------%

%--% Filósofos Comensales %-----------------------------------%
\begin{frame}
	\frametitle{\problemasTitle Filósofos Comensales}
	
	\begin{itemize}
		{\scriptsize
			\item Cuando un filosofo tiene hambre, intenta acceder a los palillos de su izquierda y derecha.
			\item Un filosofo necesita ambos palillos para comer.
			\item Un filósofo no puede quitarle un palillo a otro filósofo.
			\item \textbf{Semaphore}: palillos[5] = 0, sirviente = 1;
		}
	\end{itemize}
	
	\inputminted[
	frame=lines,
	framesep=2mm,
	baselinestretch=0.9,
	fontsize=\tiny,
	tabsize=4,
	obeytabs,
	]{c}{codigos/filosofos.c}
	\label{algoritmos:filosofos}
\end{frame}
%--------------------------------------------------------------------------------%




%--% Bibliografia %----------------------------------------------------------------------%
\begin{frame}[allowframebreaks]
	\frametitle{References}
	\nocite{UoL, Silberschatz, pctr, CECS}
	\bibliographystyle{amsalpha}
	\bibliography{bibliography}
\end{frame}
\end{document}