\documentclass{beamer}

%--% Paquetes %----------------------------------%
\usepackage[spanish]{babel}
\usepackage[utf8]{inputenc}
\usepackage[T1]{fontenc}
\usepackage{graphicx}
\usepackage{hyperref}
\usepackage{courier}
\usepackage{listings}
\usepackage{xcolor}
\usepackage{blindtext}
\usepackage{scrextend}
\usepackage[document]{ragged2e}
\usepackage{multicol}
\usepackage{pgfgantt}
\usepackage{minted}
\usepackage{tikz}
\usepackage{longtable}
\usepackage{algorithm}
\usepackage[noend]{algpseudocode}
\usepackage{amsmath}
\usepackage{wrapfig,lipsum,booktabs}
%------------------------------------------------%

% \usemintedstyle{emacs}

\usetikzlibrary{positioning,fit,calc}

\makeatletter
\newcommand*{\MoveFitHeight}[1]{%
	\pgfmathsetlengthmacro\fit@inner@sep{%
		\pgfkeysvalueof{/pgf/inner xsep}%
	}%
	\pgfmathsetlengthmacro\fit@text@height{%
		\tikz@text@height
	}%
	\kern-\fit@inner@sep\relax
	\raisebox{\fit@text@height}[0pt][0pt]{#1}%
}
\makeatother

\newcommand{\algTitle}{\textbf{Algoritmo}: }

\newcommand{\bigO}[1]{$O({#1})$}

\usetheme{Berlin}
%\usecolortheme{beaver}

%--% Personal Info %-----------------------------%
\title{Algoritmos de Planificación del Procesador}
\author{Victor Tortolero, 24.569.609}
\institute{
	Sistemas Operativos, FACYT
}
\date{\today}
%------------------------------------------------%


\begin{document}
\setbeamertemplate{caption}{\raggedright\insertcaption\par}

\begin{frame}
	\titlepage

\end{frame}
\author{ }


%--% Primera version del algoritmo de Dekker %------------------------------------%
\begin{frame}
\frametitle{Primera version del algoritmo de Dekker}

\begin{itemize}
	\item Primer algoritmo en resolver la exclusión mutua.
	\item Aplica la exclusión mutua de manera correcta, y la garantiza.
	\item Usa variables para controlar que hilo se ejecutara.
	\item Revisa constantemente si la sección critica esta disponible (spinlock, busy waiting),
	lo que malgasta el tiempo del procesador.
	\item Los procesos lentos atrasan a los rapidos.
\end{itemize}
\end{frame}

\begin{frame}
	\frametitle{Ejemplo de la Primera version del algoritmo de Dekker}
	\inputminted[
		frame=lines,
		framesep=2mm,
		baselinestretch=1.1,
		fontsize=\tiny,
		linenos
	]{c}{codigos/dekker1.c}
\end{frame}
%--------------------------------------------------------------------------------%

%--% Segunda version del algoritmo de Dekker %-----------------------------------%
\begin{frame}
	\frametitle{Segunda version del algoritmo de Dekker}
	
	\begin{itemize}
		\item Primer algoritmo en resolver la exclusión mutua.
		\item Aplica la exclusión mutua de manera correcta.
		\item Usa variables para controlar que hilo se ejecutara.
		\item Revisa constantemente si la sección critica esta disponible (spinlock, busy waiting),
		lo que malgasta el tiempo del procesador.
	\end{itemize}
\end{frame}

\begin{frame}
	\frametitle{Ejemplo de la Segunda version del algoritmo de Dekker}
	\inputminted[
	frame=lines,
	framesep=2mm,
	baselinestretch=0.95,
	fontsize=\tiny,
	linenos
	]{c}{codigos/dekker2.c}
\end{frame}
%--------------------------------------------------------------------------------%

%--% Tercera version del algoritmo de Dekker %-----------------------------------%
\begin{frame}
	\frametitle{Tercera version del algoritmo de Dekker}
	
	\begin{itemize}
		\item Garantiza la exclusión mutua.
		\item Es posible un deadlock (Ambos procesos encienden sus banderas simultáneamente, y ninguno saldría
		del ciclo).
	\end{itemize}	
\end{frame}

\begin{frame}
	\frametitle{Ejemplo de la Tercera version del algoritmo de Dekker}
	\inputminted[
	frame=lines,
	framesep=2mm,
	baselinestretch=0.95,
	fontsize=\tiny,
	linenos
	]{c}{codigos/dekker3.c}
\end{frame}
%--------------------------------------------------------------------------------%

%--% Cuarta version del algoritmo de Dekker %-----------------------------------%
\begin{frame}
	\frametitle{Cuarta version del algoritmo de Dekker}
	
	\begin{itemize}
		\item Es posible posponer un proceso de manera indefinida.
		\item Apaga las banderas por cortos periodos de tiempo para tomar control.
	\end{itemize}	
\end{frame}

\begin{frame}
	\frametitle{Ejemplo de la Cuarta version del algoritmo de Dekker}
	\inputminted[
	frame=lines,
	framesep=2mm,
	baselinestretch=0.95,
	fontsize=\tiny,
	linenos
	]{c}{codigos/dekker4.c}
\end{frame}
%--------------------------------------------------------------------------------%

%--% Quinta version del algoritmo de Dekker %-----------------------------------%
\begin{frame}
	\frametitle{Quinta version del algoritmo de Dekker}
	
	\begin{itemize}
		\item Marca procesos como preferidos para determinar el uso de las secciones criticas.
		\item El estatus de ``Preferido'' se turna entre los procesos.
		\item Garantiza la exclusión mutua.
		\item Evita deadlock's, y el posponer un proceso de manera indefinida.
	\end{itemize}	
\end{frame}

\begin{frame}[allowframebreaks]
	\frametitle{Ejemplo de la Quinta version del algoritmo de Dekker}
	\inputminted[
	frame=lines,
	framesep=2mm,
	baselinestretch=1.1,
	fontsize=\tiny,
	linenos
	]{c}{codigos/dekker5.c}
\end{frame}
%--------------------------------------------------------------------------------%

%--% Test and Set %-----------------------------------%
\begin{frame}
	\frametitle{Test and Set}
	
	\begin{itemize}
		\item Operación Atómica.
		\item Retorna el valor del lock, y lo cambia a verdadero.
		\item Si el valor retornado es \textbf{falso}, obtenemos el lock. Si es \textbf{verdadero}, 
		esta ocupado por otro proceso.
	\end{itemize}	
	
	\inputminted[
	frame=lines,
	framesep=2mm,
	baselinestretch=1.0,
	fontsize=\tiny,
	linenos
	]{c}{codigos/test_and_set.c}
\end{frame}
%--------------------------------------------------------------------------------%

%--% Compare and Swap %-----------------------------------%
\begin{frame}
	\frametitle{Compare and Swap}
	
	\begin{itemize}
		\item Operación Atómica.
		\item Retorna el valor del lock, y lo cambia a verdadero.
		\item Si el valor retornado es \textbf{falso}, obtenemos el lock. Si es \textbf{verdadero}, 
		esta ocupado por otro proceso.
	\end{itemize}	
	
	\inputminted[
	frame=lines,
	framesep=2mm,
	baselinestretch=1.0,
	fontsize=\tiny,
	linenos
	]{c}{codigos/compare_and_swap.c}
\end{frame}
%--------------------------------------------------------------------------------%


%--% Bibliografia %----------------------------------------------------------------------%
\begin{frame}[allowframebreaks]
	\frametitle{References}
	\nocite{UoL, Silberschatz}
	\bibliographystyle{amsalpha}
	\bibliography{bibliography}
\end{frame}
\end{document}