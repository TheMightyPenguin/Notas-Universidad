\documentclass{beamer}

%--% Paquetes %----------------------------------%
\usepackage[spanish]{babel}
\usepackage[utf8]{inputenc}
\usepackage[T1]{fontenc}
\usepackage{graphicx}
\usepackage{hyperref}
\usepackage{courier}
\usepackage{listings}
\usepackage{xcolor}
\usepackage{blindtext}
\usepackage{scrextend}
\usepackage[document]{ragged2e}
\usepackage{multicol}
\usepackage{pgfgantt}
\usepackage{minted}
\usepackage{tikz}
\usepackage{longtable}
\usepackage{algorithm}
\usepackage[noend]{algpseudocode}
\usepackage{amsmath}
\usepackage{wrapfig,lipsum,booktabs}
%------------------------------------------------%

%En caso de que LaTeX separe las palabras con - de manera incorrecta, usar
%\hyphenation{deci-sión,e-xa-men, otras palabras....}
\hyphenation{o-cu-rrir}


\usetikzlibrary{positioning,fit,calc}

\makeatletter
\newcommand*{\MoveFitHeight}[1]{%
	\pgfmathsetlengthmacro\fit@inner@sep{%
		\pgfkeysvalueof{/pgf/inner xsep}%
	}%
	\pgfmathsetlengthmacro\fit@text@height{%
		\tikz@text@height
	}%
	\kern-\fit@inner@sep\relax
	\raisebox{\fit@text@height}[0pt][0pt]{#1}%
}
\makeatother

\newcommand{\algTitle}{\textbf{Algoritmo:} }

\newcommand{\bigO}[1]{$O({#1})$}

\usetheme{Berlin}
%\usecolortheme{beaver}

%--% Personal Info %-----------------------------%
\title{Almacenamiento Intermedio de E/S}
\author{Victor Tortolero, 24.569.609}
\institute{
	Sistemas Operativos, FACYT
}
\date{\today}
%------------------------------------------------%

\begin{document}
\setbeamertemplate{caption}{\raggedright\insertcaption\par}

\begin{frame}
	\titlepage
\end{frame}


\begin{frame}
	\frametitle{Almacenamiento Intermedio de E/S}
	
	Espacio de memoria principal que se reserva para el almacenamiento intermedio de datos procedentes o con destino a los periféricos.
	
\end{frame}

\begin{frame}
	\frametitle{Almacenamiento Intermedio de E/S}
	\begin{itemize}
		\item \textbf{E/S sin buffer}: La transferencia se realiza directamente sobre el buffer de usuario.
		\item \textbf{Buffer simple}: La transferencia de un bloque de la entrada se hace desde el dispositivo al
		buffer que el s.o. le reserva en la memoria principal.
	\end{itemize}
\end{frame}
	
\begin{frame}
	\frametitle{Almacenamiento Intermedio de E/S}
	\begin{itemize}
		\item \textbf{Buffer doble}: Un proceso transfiere datos a (o desde) uno de los buffers mientras el s.o. vacía (o llena) el otro buffer.
		\item \textbf{Buffer circular}: Permite soportar concurrencia del tipo productor-consumidor.
	\end{itemize}
\end{frame}

\end{document}

