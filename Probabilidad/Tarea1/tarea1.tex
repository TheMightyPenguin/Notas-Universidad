\documentclass{article}

\usepackage[spanish]{babel}
\usepackage[utf8]{inputenc}
\usepackage[T1]{fontenc}
\usepackage{graphicx}
\usepackage{wrapfig}
\usepackage{pdfpages}
\usepackage{hyperref}
\usepackage{courier}
\usepackage{longtable}
\usepackage{listings}
\usepackage{minted}
\usepackage{xcolor}
\usepackage{blindtext}
\usepackage{scrextend}
\usepackage[document]{ragged2e}
\usepackage{multicol}
\usepackage{booktabs}
\usepackage{amsmath}
\usepackage{amssymb}
 % % % % % % % %

\usemintedstyle{pastie}

\usepackage{array}
\newcolumntype{L}[1]{>{\raggedright\let\newline\\\arraybackslash\hspace{0pt}}m{#1}}
\newcolumntype{C}[1]{>{\centering\let\newline\\\arraybackslash\hspace{0pt}}m{#1}}
\newcolumntype{R}[1]{>{\raggedleft\let\newline\\\arraybackslash\hspace{0pt}}m{#1}}

%Custom commands
\newcommand\myeq{\mathrel{\overset{\makebox[0pt]{\mbox{\normalfont\tiny\sffamily 24 Ceros}}}{00\dots00}}}

\usepackage{anysize}
\marginsize{2.54cm}{2.54cm}{2.54cm}{2.54cm}

\usepackage{setspace}
\onehalfspacing
%\doublespacing

\setlength{\columnsep}{1cm}

%En caso de que LaTeX separe las palabras con - de manera incorrecta, usar
%\hyphenation{deci-sión,e-xa-men, otras palabras....}

\setlength{\columnseprule}{2pt}
\def\columnseprulecolor{\color{black}}


%--% Comandos personalizados %--%
\newcommand{\bigO}[1]{$O({#1})$}

\newcommand{\set}[1]{\left\{{#1}\right\}}


%-------------------------------%

\begin{document}

\flushleft
\setlength{\parindent}{20pt}

\justify
%%CUERPO PRINCIPAl%%%%%%%%%%%%%%%%%%%%%%%%%%%%%%%%%%%%%%%%%%%
\centerline{\huge Tarea 1 \textbf{Probabilidad}}
\centerline{Victor Tortolero CI:24.569.609}  % Pon tu nombre y Cedula!
\vspace{0.1cm}
\hrule


%--% Respuesta a %---------------------------------------------------------%
\section*{a.1)Demuestre que $A \cap B = B \cap A$}

Tenemos $\underbrace{A \cap B}_{L1} =\underbrace{ B \cap A}_{L2}$, partiremos de $L1$ para llegar a $L2$.

\begin{align*}
\text{Sea $x$ un elemento cualquiera:} \\
x \in (A \cap B) & \Leftrightarrow x \in A \land x \in B & \text{Definición de la intersección} \\
                 & \Leftrightarrow x \in B \land x \in A & \text{Conmutativa} \\
                 & \Leftrightarrow x \in (B \cap A) & \text{Definición de la intersección}
\end{align*}
$\therefore$ de $L1$ llegamos a $L2$ aplicando leyes lógicas y propiedades de conjuntos, y
	por lo tanto queda demostrado que $A \cap B = B \cap A$

%-- -- -- -- -- --%

\section*{a.2)Demuestre que $A \cup B = B \cup A$}
	
Tenemos $\underbrace{A \cup B}_{L1} =\underbrace{ B \cup A}_{L2}$, partiremos de $L1$ para llegar a $L2$.

\begin{align*}
\text{Sea $x$ un elemento cualquiera:} \\
x \in (A \cup B) & \Leftrightarrow x \in A \lor x \in B & \text{Definición de la unión} \\
                 & \Leftrightarrow x \in B \lor x \in A & \text{Conmutativa} \\
                 & \Leftrightarrow x \in (B \cup A) & \text{Definición de la unión}
\end{align*}
$\therefore$ de $L1$ llegamos a $L2$ aplicando leyes lógicas y propiedades de conjuntos, y
por lo tanto queda demostrado que $A \cup B = B \cup A$

\newpage
%--------------------------------------------------------------------------%	


%--% Respuesta b %---------------------------------------------------------%
\section*{b.1)Demuestre que $A \cup (B \cup C) = (A \cup B) \cup C$}

Tenemos $\underbrace{A \cup (B \cup C)}_{L1} =\underbrace{(A \cup B) \cup C}_{L2}$, partiremos de $L1$ para llegar a $L2$.

\begin{align*}
\text{Sea $x$ un elemento cualquiera:} \\
x \in (A \cup (B \cup C)) & \Leftrightarrow x \in A \lor x \in (B \cup C) & \text{Definición de la unión} \\
                          & \Leftrightarrow x \in A \lor (x \in B \lor x \in C) & \text{Definicion de la unión} \\
                          & \Leftrightarrow (x \in A \lor x \in B) \lor x \in C & \text{Asociatividad} \\
                          & \Leftrightarrow x \in (A \cup B) \lor x \in C & \text{Definición de la unión} \\
                          & \Leftrightarrow x \in ((A \cup B) \cup C) & \text{Definición de la unión} \\
\end{align*}
$\therefore$ de $L1$ llegamos a $L2$ aplicando leyes lógicas y propiedades de conjuntos, y
por lo tanto queda demostrado que $A \cup (B \cup C) = (A \cup B) \cup C$

%-- -- -- -- -- --%

\newpage
%--------------------------------------------------------------------------%	

\end{document}
