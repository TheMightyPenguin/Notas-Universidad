% Pregunta 1

\section*{Pregunta 1}

1.a Queremos calcular la probabilidad de obtener al menos un 6 en 4 lanzamientos de un dado de 6 caras, nuestro espacio muestral $S$ serian todas las posibles permutaciones al lanzar 4 veces un dado de 6 caras, y estamos interesados en el evento $A$, que es obtener al menos un 6 al lanzar 4 veces un dado de seis caras, entonces tenemos que:
\begin{align*}
	\#S = 6^4 = 1296
\end{align*}
Luego para calcular $\prob{A}$, necesitamos saber cuantos elementos contiene $A$, y para esto podemos restarle a la cantidad de elementos totales en el caso que tuvieramos un dado de 5 caras y lo lanzáramos 4 veces, entonces:
\begin{align*}
	\prob{A} = \frac{1296 - 54}{1296} \approx 0.5177 = 51\%
\end{align*}


1.b Queremos calcular la probabilidad de obtener un doble seis en una seria de 24 lanzamientos de un dado de 6 caras,
