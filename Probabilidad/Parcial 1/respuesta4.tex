% Pregunta 4

\section*{Pregunta 4}

Una aseguradora tiene clientes de riesgo alto, medio y bajo. Estos clientes tienen probabilidades de 0.02, 0.01 y 0.0025 de rellenar un impreso de reclamación. Si la proporción de clientes de alto riesgo es 0.1, de riesgo medio 0.2 y de bajo riesgo 0.7. ?`Cuál es la probabilidad de que un impreso rellenado sea de un cliente de alto riesgo?

\vspace{0.40cm}
\textbf{R}: Definimos los siguientes eventos:
\begin{align*}
& A: \text{es cliente de alto riesgo} \hspace{0.7cm} B: \text{es cliente de medio riesgo} \hspace{0.7cm} C: \text{es cliente de bajo riesgo} \\
& D: \text{el cliente rellena un impreso de reclamacion}
\end{align*}
Entonces, por el teorema de Bayes podemos calcular la probabilidad de que un impreso rellanado sea de un cliente de alto riesgo.
\begin{align*}
	\prob{A | D} & = \frac{\prob{A} \prob{D | A}}{\prob{A} \prob{D | A} 
	+ \prob{B} \prob{D | B} + \prob{C} \prob{D | C}} \\
	& = \frac{(0.1)(0.02)}{(0.1)(0.02) + (0.2)(0.01) + (0.7)(0.0025)}\\
	& \approx 0.3478 
\end{align*}
Entonces, la probabilidad condicional de que un cliente sea de alto riesgo si relleno un impreso de reclamo, es de $34.78\%$.
