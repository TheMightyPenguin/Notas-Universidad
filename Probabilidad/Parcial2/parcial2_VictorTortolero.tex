\documentclass{article}

\usepackage[spanish]{babel}
\usepackage[utf8]{inputenc}
\usepackage[T1]{fontenc}
\usepackage{graphicx}
\usepackage{wrapfig}
\usepackage{pdfpages}
\usepackage{hyperref}
\usepackage{courier}
\usepackage{capt-of}
\usepackage{float}
\usepackage{longtable}
\usepackage{standalone}
\usepackage{listings}
\usepackage{enumitem}
\usepackage{minted}
\usepackage{xcolor}
\usepackage{blindtext}
\usepackage{scrextend}
\usepackage{pgfplots}
\usepackage[document]{ragged2e}
\usepackage{multicol}
\usepackage{booktabs}
\usepackage{amsmath}
\usepackage{breqn}
\usepackage{tkz-fct}  
\usepackage{lmodern}
\usepackage{amsmath}
\usepackage{amssymb}
 % % % % % % % %

% \setmainfont{Calibri}

\usemintedstyle{pastie}
\decimalpoint

\usepackage{array}
\newcolumntype{L}[1]{>{\raggedright\let\newline\\\arraybackslash\hspace{0pt}}m{#1}}
\newcolumntype{C}[1]{>{\centering\let\newline\\\arraybackslash\hspace{0pt}}m{#1}}
\newcolumntype{R}[1]{>{\raggedleft\let\newline\\\arraybackslash\hspace{0pt}}m{#1}}

%Custom commands
\newcommand\myeq{\mathrel{\overset{\makebox[0pt]{\mbox{\normalfont\tiny\sffamily 24 Ceros}}}{00\dots00}}}

\usepackage{anysize}
\marginsize{1.2cm}{1.2cm}{1.2cm}{1.2cm}

\usepackage{setspace}
\onehalfspacing
%\doublespacing

\setlength{\columnsep}{1cm}

%En caso de que LaTeX separe las palabras con - de manera incorrecta, usar
%\hyphenation{deci-sión,e-xa-men, otras palabras....}

\setlength{\columnseprule}{2pt}
\def\columnseprulecolor{\color{black}}

\newcommand{\prob}[1]{{P} \left( {#1} \right)}

\newcommand{\dist}[1]{{p} \left( {#1} \right)}


% Aqui comienza el documento como tal!!
\begin{document}

\flushleft
\setlength{\parindent}{20pt}

\justify
%%CUERPO PRINCIPAl%%%%%%%%%%%%%%%%%%%%%%%%%%%%%%%%%%%%%%%%%%%
\centerline{\huge Parcial 2 \textbf{Probabilidad}}
\centerline{Victor Tortolero CI:24.569.609}  % Pon tu nombre y Cedula!
\vspace{0.1cm}
\hrule


%--------------------------------------------------------------%

% respuesta1.tex

2. Normalice hasta la 3NF y muestre cómo es la BCNF - en caso de que ésta sea diferente a la 3NF - para la siguiente relación. Debe justificar cada paso de normalización.

Produccion\_de\_Vegetales(\underline{v\#}, sembradio, año, calidad, región, país, \underline{cip}, nombrep, \underline{fecha}, cantidad)

Produccion\_de\_Vegetales(v\#) $\rightarrow$ Produccion\_de\_Vegetales(país)

Produccion\_de\_Vegetales(v\#) $\rightarrow$ Produccion\_de\_Vegetales(sembradio)

Produccion\_de\_Vegetales(v\#) $\rightarrow$ Produccion\_de\_Vegetales(año)

Produccion\_de\_Vegetales(v\#) $\rightarrow$ Produccion\_de\_Vegetales(calidad)

Produccion\_de\_Vegetales(v\#) $\rightarrow$ Produccion\_de\_Vegetales(región)

Produccion\_de\_Vegetales(sembradio,año) $\rightarrow$ Produccion\_de\_Vegetales(calidad)

Produccion\_de\_Vegetales(región) $\rightarrow$ Produccion\_de\_Vegetales(pais)

Produccion\_de\_Vegetales(cip) $\rightarrow$ Produccion\_de\_Vegetales(nombrep)

Produccion\_de\_Vegetales(v\#,cip,fecha) $\rightarrow$ Produccion\_de\_Vegetales(cantidad)


\subsection*{1NF}
Ya se encuentra en 1NF ya que todos los atributos de la relación contienen valores atómicos.

\begin{figure}[H]
	\centering
	\includegraphics[scale=0.8]{img/1_1nf.png}
\end{figure}

\newpage
\subsection*{2NF}
Tenemos que sembradio, año, calidad, región, país, y nombrep no dependen totalmente de la clave primaria, por lo que procedemos a separar la relación, y tendríamos:

Almacen(\underline{v\#}, \underline{cip}, \underline{fecha}, cantidad)

Produccion(\underline{v\#}, sembradio, año, calidad, region, pais)

Nombres(\underline{cip}, nombrep)

\begin{figure}[H]
	\centering 
	\begin{overpic}[scale=0.56,]{img/2nf_almacen.png}
		\put (35, 65) {\textbf{Almacen}}
	\end{overpic}
	\hspace{0.4cm} \vrule \hspace{0.4cm}
	\begin{overpic}[scale=0.56,]{img/2nf_produccion.png}
		\put (40, 65) {\textbf{Producción}}
	\end{overpic}
	\hspace{0.4cm} \vrule \hspace{0.4cm}
	\begin{overpic}[scale=0.56,]{img/2nf_nombres.png}
		\put (35, 35) {\textbf{Nombres}}
	\end{overpic}
\end{figure}


\subsection*{3NF}
Tenemos que calidad y pais no dependen solo de la clave primaria y tienen dependencias funcionales transitivas, y esto viola la 3NF, por lo que pasamos estos atributos a otras relaciones:

Almacen(\underline{v\#}, \underline{cip}, \underline{fecha}, cantidad)

Produccion(\underline{v\#}, sembradio, año, región)

Siembra(\underline{sembradio}, \underline{año}, calidad)

Lugar(\underline{región}, país)

Nombre(\underline{cip}, nombrep)

\vspace{0.40cm}
\begin{figure}[H]
	\centering 
	\begin{overpic}[scale=0.60,]{img/2nf_almacen.png}
		\put (35, 65) {\textbf{Almacen}}
	\end{overpic}
	\hspace{0.4cm} \vrule \hspace{0.4cm}
	\begin{overpic}[scale=0.60,]{img/1-3nf_produccion.png}
		\put (40, 74) {\textbf{Producción}}
	\end{overpic}
	\hspace{0.4cm} \vrule \hspace{0.4cm}
	\begin{overpic}[scale=0.56,]{img/1-3nf_siembra.png}
		\put (35, 52) {\textbf{Siembra}}
	\end{overpic}
\end{figure}

\hrule
\vspace{0.40cm} 
\begin{figure}[H]
	\centering 
	\begin{overpic}[scale=0.60,]{img/1-3nf_lugar.png}
		\put (35, 30) {\textbf{Lugar}}
	\end{overpic}
	\hspace{0.4cm} \vrule \hspace{0.4cm}
	\begin{overpic}[scale=0.60,]{img/2nf_nombres.png}
		\put (40, 30) {\textbf{Nombre}}
	\end{overpic}
\end{figure}

\subsection*{BCNF}
Como ningún atributo no clave es determinante, tenemos que nuestras relaciones se encuentran en BCNF. \newpage

% respuesta2.tex

El método de Gauss-Jordan 
\[
\begin{array}{ccccccc}
	a_{11}x_1 & + & a_{12}x_2 & + & \dots & + & a_{1n}x_n \\
	a_{21}x_1 & + & a_{22}x_2 & + & \dots & + & a_{2n}x_n \\
	\vdots &  & \vdots &  & \ddots &  & \vdots \\
	a_{n1}x_1 & + & a_{n2}x_2 & + & \dots & + & a_{nn}x_n
\end{array}
\] \newpage

% respuesta3.tex

Al correr el programa, para encontrar las primeras 10 raíces positivas de la ecuación:
\begin{equation*}
	x - \tan{x} = 0
\end{equation*}
obtenemos la siguiente información: 

\begin{figure}[H]
	\centering
	\caption{Punto inicial: 4.66}
	\begin{tabular}{|c|c|c|c|} \hline
		$i$ & $x_{i}$ & $f(x_{i})$ & $f'(x_{i})$ \\ \hline
		$0$ & $4.6623890$ & $-15.3209416$ & $-399.3335001$ \\ \hline
		$1$ & $4.6240227$ & $-6.6630402$ & $-127.3977882$ \\ \hline
		$2$ & $4.5717216$ & $-2.4902976$ & $-49.8721154$ \\ \hline
		$3$ & $4.5217880$ & $-0.6610857$ & $-26.8621799$ \\ \hline
		$4$ & $4.4971777$ & $-0.0774590$ & $-20.9273010$ \\ \hline
		$5$ & $4.4934763$ & $-0.0013510$ & $-20.2034728$ \\ \hline
		$6$ & $4.4934095$ & $-0.0000004$ & $-20.1907326$ \\ \hline
		$7$ & $4.4934095$ & $-0.0000000$ & $-20.1907286$ \\ \hline
	\end{tabular}
\end{figure}
\begin{figure}[H]
	\centering
	\caption{Punto inicial: 7.80}
	\begin{tabular}{|c|c|c|c|} \hline
		$i$ & $x_{i}$ & $f(x_{i})$ & $f'(x_{i})$ \\ \hline
		$0$ & $7.8039816$ & $-12.1793489$ & $-399.3335001$ \\ \hline
		$1$ & $7.7734824$ & $-4.6221579$ & $-153.6518994$ \\ \hline
		$2$ & $7.7434004$ & $-1.2628375$ & $-81.1123214$ \\ \hline
		$3$ & $7.7278314$ & $-0.1571323$ & $-62.1726535$ \\ \hline
		$4$ & $7.7253041$ & $-0.0031192$ & $-59.7285267$ \\ \hline
		$5$ & $7.7252519$ & $-0.0000013$ & $-59.6795360$ \\ \hline
		$6$ & $7.7252518$ & $-0.0000000$ & $-59.6795159$ \\ \hline
	\end{tabular}
\end{figure}
\begin{figure}[H]
	\centering
	\caption{Punto inicial: 10.95}
	\begin{tabular}{|c|c|c|c|} \hline
		$i$ & $x_{i}$ & $f(x_{i})$ & $f'(x_{i})$ \\ \hline
		$0$ & $10.9455743$ & $-9.0377563$ & $-399.3335001$ \\ \hline
		$1$ & $10.9229422$ & $-2.8208555$ & $-188.8919748$ \\ \hline
		$2$ & $10.9080085$ & $-0.4827718$ & $-129.7498756$ \\ \hline
		$3$ & $10.9042877$ & $-0.0197784$ & $-119.3352211$ \\ \hline
		$4$ & $10.9041220$ & $-0.0000360$ & $-118.9006618$ \\ \hline
		$5$ & $10.9041217$ & $-0.0000000$ & $-118.8998692$ \\ \hline
	\end{tabular}
\end{figure}
\begin{figure}[H]
	\centering
	\caption{Punto inicial: 14.09}
	\begin{tabular}{|c|c|c|c|} \hline
		$i$ & $x_{i}$ & $f(x_{i})$ & $f'(x_{i})$ \\ \hline
		$0$ & $14.0871669$ & $-5.8961636$ & $-399.3335001$ \\ \hline
		$1$ & $14.0724019$ & $-1.3464395$ & $-237.7406718$ \\ \hline
		$2$ & $14.0667384$ & $-0.1085768$ & $-200.9395614$ \\ \hline
		$3$ & $14.0661981$ & $-0.0008294$ & $-197.8812635$ \\ \hline
		$4$ & $14.0661939$ & $-0.0000000$ & $-197.8578126$ \\ \hline
	\end{tabular}
\end{figure}
\begin{figure}[H]
	\centering
	\caption{Punto inicial: 17.23}
	\begin{tabular}{|c|c|c|c|} \hline
		$i$ & $x_{i}$ & $f(x_{i})$ & $f'(x_{i})$ \\ \hline
		$0$ & $17.2287596$ & $-2.7545710$ & $-399.3335001$ \\ \hline
		$1$ & $17.2218617$ & $-0.3345028$ & $-308.2259352$ \\ \hline
		$2$ & $17.2207764$ & $-0.0062743$ & $-296.7712770$ \\ \hline
		$3$ & $17.2207553$ & $-0.0000023$ & $-296.5544913$ \\ \hline
	\end{tabular}
\end{figure}
\begin{figure}[H]
	\centering
	\caption{Punto inicial: 20.37}
	\begin{tabular}{|c|c|c|c|} \hline
		$i$ & $x_{i}$ & $f(x_{i})$ & $f'(x_{i})$ \\ \hline
		$0$ & $20.3703522$ & $0.3870217$ & $-399.3335001$ \\ \hline
		$1$ & $20.3713214$ & $-0.0076628$ & $-415.3029993$ \\ \hline
		$2$ & $20.3713030$ & $-0.0000029$ & $-414.9901022$ \\ \hline
	\end{tabular}
\end{figure}
\begin{figure}[H]
	\centering
	\caption{Punto inicial: 23.51}
	\begin{tabular}{|c|c|c|c|} \hline
		$i$ & $x_{i}$ & $f(x_{i})$ & $f'(x_{i})$ \\ \hline
		$0$ & $23.5119449$ & $3.5286143$ & $-399.3335001$ \\ \hline
		$1$ & $23.5207812$ & $-0.7587207$ & $-589.4942094$ \\ \hline
		$2$ & $23.5194941$ & $-0.0230297$ & $-554.2504282$ \\ \hline
		$3$ & $23.5194525$ & $-0.0000225$ & $-553.1657083$ \\ \hline
		$4$ & $23.5194525$ & $-0.0000000$ & $-553.1646458$ \\ \hline
	\end{tabular}
\end{figure}
\begin{figure}[H]
	\centering
	\caption{Punto inicial: 26.65}
	\begin{tabular}{|c|c|c|c|} \hline
		$i$ & $x_{i}$ & $f(x_{i})$ & $f'(x_{i})$ \\ \hline
		$0$ & $26.6535376$ & $6.6702070$ & $-399.3335001$ \\ \hline
		$1$ & $26.6702409$ & $-3.3517103$ & $-901.3175525$ \\ \hline
		$2$ & $26.6665222$ & $-0.3369726$ & $-729.1887304$ \\ \hline
		$3$ & $26.6660601$ & $-0.0041589$ & $-711.3005805$ \\ \hline
		$4$ & $26.6660543$ & $-0.0000006$ & $-711.0784844$ \\ \hline
	\end{tabular}
\end{figure}
\begin{figure}[H]
	\centering
	\caption{Punto inicial: 29.80}
	\begin{tabular}{|c|c|c|c|} \hline
		$i$ & $x_{i}$ & $f(x_{i})$ & $f'(x_{i})$ \\ \hline
		$0$ & $29.7951302$ & $9.8117997$ & $-399.3335001$ \\ \hline
		$1$ & $29.8197006$ & $-9.4961358$ & $-1545.7349954$ \\ \hline
		$2$ & $29.8135572$ & $-1.8485454$ & $-1002.4887435$ \\ \hline
		$3$ & $29.8117132$ & $-0.1020710$ & $-894.8344894$ \\ \hline
		$4$ & $29.8115992$ & $-0.0003475$ & $-888.7521642$ \\ \hline
		$5$ & $29.8115988$ & $-0.0000000$ & $-888.7314227$ \\ \hline
	\end{tabular}
\end{figure}
\begin{figure}[H]
	\centering
	\caption{Punto inicial: 32.94}
	\begin{tabular}{|c|c|c|c|} \hline
		$i$ & $x_{i}$ & $f(x_{i})$ & $f'(x_{i})$ \\ \hline
		$0$ & $32.9367229$ & $12.9533923$ & $-399.3335001$ \\ \hline
		$1$ & $32.9691604$ & $-23.9645837$ & $-3241.4512185$ \\ \hline
		$2$ & $32.9617672$ & $-7.1010237$ & $-1605.0272201$ \\ \hline
		$3$ & $32.9573430$ & $-1.0697688$ & $-1157.8443361$ \\ \hline
		$4$ & $32.9564191$ & $-0.0326349$ & $-1088.2776818$ \\ \hline
		$5$ & $32.9563891$ & $-0.0000323$ & $-1086.1257083$ \\ \hline
		$6$ & $32.9563890$ & $-0.0000000$ & $-1086.1235785$ \\ \hline
	\end{tabular}
\end{figure}
\vspace{1cm}

Estas serian las 10 primeras raíces positivas para la ecuación:
\begin{enumerate}
	\item $4.4934095$
	\item $7.7252518$
	\item $10.9041217$
	\item $14.0661939$
	\item $17.2207553$
	\item $20.3713030$
	\item $23.5194525$
	\item $26.6660543$
	\item $29.8115988$
	\item $32.9563890$
\end{enumerate}
 \newpage


\end{document}
