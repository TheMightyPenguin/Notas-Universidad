\documentclass{article}

\usepackage[spanish]{babel}
\usepackage[utf8]{inputenc}
\usepackage[T1]{fontenc}
\usepackage{graphicx}
\usepackage{wrapfig}
\usepackage{pdfpages}
\usepackage{hyperref}
\usepackage{courier}
\usepackage{capt-of}
\usepackage{float}
\usepackage{longtable}
\usepackage{standalone}
\usepackage{listings}
\usepackage{enumitem}
\usepackage{minted}
\usepackage{xcolor}
\usepackage{blindtext}
\usepackage{scrextend}
\usepackage{pgfplots}
\usepackage[document]{ragged2e}
\usepackage{multicol}
\usepackage{booktabs}
\usepackage{amsmath}
\usepackage{breqn}
\usepackage{tkz-fct}  
\usepackage{lmodern}
\usepackage{amsmath}
\usepackage{amssymb}
 % % % % % % % %

% \setmainfont{Calibri}

\usemintedstyle{pastie}
\decimalpoint

\usepackage{array}
\newcolumntype{L}[1]{>{\raggedright\let\newline\\\arraybackslash\hspace{0pt}}m{#1}}
\newcolumntype{C}[1]{>{\centering\let\newline\\\arraybackslash\hspace{0pt}}m{#1}}
\newcolumntype{R}[1]{>{\raggedleft\let\newline\\\arraybackslash\hspace{0pt}}m{#1}}

%Custom commands
\newcommand\myeq{\mathrel{\overset{\makebox[0pt]{\mbox{\normalfont\tiny\sffamily 24 Ceros}}}{00\dots00}}}

\usepackage{anysize}
\marginsize{1.2cm}{1.2cm}{1.2cm}{1.2cm}

\usepackage{setspace}
\onehalfspacing
%\doublespacing

\setlength{\columnsep}{1cm}

%En caso de que LaTeX separe las palabras con - de manera incorrecta, usar
%\hyphenation{deci-sión,e-xa-men, otras palabras....}

\setlength{\columnseprule}{2pt}
\def\columnseprulecolor{\color{black}}

\newcommand{\prob}[1]{{P} \left( {#1} \right)}

\newcommand{\dist}[1]{{p} \left( {#1} \right)}


% Aqui comienza el documento como tal!!
\begin{document}

\flushleft
\setlength{\parindent}{20pt}

\justify
%%CUERPO PRINCIPAl%%%%%%%%%%%%%%%%%%%%%%%%%%%%%%%%%%%%%%%%%%%
\centerline{\huge Parcial 2 \textbf{Probabilidad}}
\centerline{Victor Tortolero CI:24.569.609}  % Pon tu nombre y Cedula!
\vspace{0.1cm}
\hrule


%--------------------------------------------------------------%

% respuesta1.tex

Al correr el programa, que calcula el resultado del método de Newton en para hallar raíces de la función:

\begin{equation*}
	f(t) = 1 - \dfrac{3}{2} * e^{(-1 * (t*t - 2*t + 0.9775))} - e^{(-1 * (t*t - 8*t + 15.96))}
\end{equation*}
con los siguientes puntos iniciales:
\begin{equation*}
[0.80, 0.85, 0.90, 0.95, 1.00, 1.05, 1.10, 1.15, 1.20]
\end{equation*}
obtenemos la siguiente información: 

\begin{figure}[H]
	\centering
	\caption{Punto inicial: 0.80}
	\begin{tabular}{|c|c|c|c|} \hline
		$i$ & $x_{i}$ & $f(x_{i})$ & $f'(x_{i})$ \\ \hline
		$0$ & $0.8000000$ & $-0.4740155$ & $-0.5898284$ \\ \hline
		$1$ & $-0.0036499$ & $0.4397364$ & $-1.1246161$ \\ \hline
		$2$ & $0.3873603$ & $-0.0540499$ & $-1.2915172$ \\ \hline
		$3$ & $0.3455103$ & $0.0003899$ & $-1.3084770$ \\ \hline
		$4$ & $0.3458083$ & $0.0000000$ & $-1.3083914$ \\ \hline
	\end{tabular}
\end{figure}
\begin{figure}[H]
	\centering
	\caption{Punto inicial: 0.85}
	\begin{tabular}{|c|c|c|c|} \hline
		$i$ & $x_{i}$ & $f(x_{i})$ & $f'(x_{i})$ \\ \hline
		$0$ & $0.8500000$ & $-0.5000511$ & $-0.4503210$ \\ \hline
		$1$ & $-0.2604324$ & $0.6867409$ & $-0.7896827$ \\ \hline
		$2$ & $0.6092092$ & $-0.3168673$ & $-1.0293012$ \\ \hline
		$3$ & $0.3013622$ & $0.0583564$ & $-1.3157409$ \\ \hline
		$4$ & $0.3457147$ & $0.0001225$ & $-1.3084183$ \\ \hline
		$5$ & $0.3458083$ & $0.0000000$ & $-1.3083914$ \\ \hline
	\end{tabular}
\end{figure}
\begin{figure}[H]
	\centering
	\caption{Punto inicial: 0.90}
	\begin{tabular}{|c|c|c|c|} \hline
		$i$ & $x_{i}$ & $f(x_{i})$ & $f'(x_{i})$ \\ \hline
		$0$ & $0.9000000$ & $-0.5189375$ & $-0.3042058$ \\ \hline
		$1$ & $-0.8058763$ & $0.9411771$ & $-0.2124534$ \\ \hline
		$2$ & $3.6241634$ & $0.0947282$ & $-0.6710567$ \\ \hline
		$3$ & $3.7653262$ & $0.0142263$ & $-0.4582679$ \\ \hline
		$4$ & $3.7963698$ & $0.0008478$ & $-0.4032092$ \\ \hline
		$5$ & $3.7984726$ & $0.0000040$ & $-0.3993906$ \\ \hline
		$6$ & $3.7984826$ & $0.0000000$ & $-0.3993724$ \\ \hline
	\end{tabular}
\end{figure}
\begin{figure}[H]
	\centering
	\caption{Punto inicial: 0.95}
	\begin{tabular}{|c|c|c|c|} \hline
		$i$ & $x_{i}$ & $f(x_{i})$ & $f'(x_{i})$ \\ \hline
		$0$ & $0.9500000$ & $-0.5303969$ & $-0.1536090$ \\ \hline
		$1$ & $-2.5029032$ & $0.9999928$ & $-0.0000504$ \\ \hline
	\end{tabular}
\end{figure}
\begin{figure}[H]
	\centering
	\caption{Punto inicial: 1.00}
	\begin{tabular}{|c|c|c|c|} \hline
		$i$ & $x_{i}$ & $f(x_{i})$ & $f'(x_{i})$ \\ \hline
		$0$ & $1.0000000$ & $-0.5342610$ & $-0.0007707$ \\ \hline
	\end{tabular}
\end{figure}
\begin{figure}[H]
	\centering
	\caption{Punto inicial: 1.05}
	\begin{tabular}{|c|c|c|c|} \hline
		$i$ & $x_{i}$ & $f(x_{i})$ & $f'(x_{i})$ \\ \hline
		$0$ & $1.0500000$ & $-0.5304750$ & $0.1520096$ \\ \hline
		$1$ & $4.5397473$ & $0.2222265$ & $0.8396434$ \\ \hline
		$2$ & $4.2750795$ & $0.0350064$ & $0.5311111$ \\ \hline
		$3$ & $4.2091679$ & $0.0036925$ & $0.4171101$ \\ \hline
		$4$ & $4.2003153$ & $0.0000715$ & $0.4009391$ \\ \hline
		$5$ & $4.2001369$ & $0.0000000$ & $0.4006113$ \\ \hline
		$6$ & $4.2001368$ & $0.0000000$ & $0.4006112$ \\ \hline
	\end{tabular}
\end{figure}
\begin{figure}[H]
	\centering
	\caption{Punto inicial: 1.10}
	\begin{tabular}{|c|c|c|c|} \hline
		$i$ & $x_{i}$ & $f(x_{i})$ & $f'(x_{i})$ \\ \hline
		$0$ & $1.1000000$ & $-0.5190994$ & $0.3024291$ \\ \hline
		$1$ & $2.8164331$ & $0.6869363$ & $-0.4013668$ \\ \hline
		$2$ & $4.5279259$ & $0.2123474$ & $0.8316886$ \\ \hline
		$3$ & $4.2726052$ & $0.0336973$ & $0.5270527$ \\ \hline
		$4$ & $4.2086698$ & $0.0034850$ & $0.4162053$ \\ \hline
		$5$ & $4.2002966$ & $0.0000640$ & $0.4009047$ \\ \hline
		$6$ & $4.2001369$ & $0.0000000$ & $0.4006113$ \\ \hline
		$7$ & $4.2001368$ & $0.0000000$ & $0.4006112$ \\ \hline
	\end{tabular}
\end{figure}
\begin{figure}[H]
	\centering
	\caption{Punto inicial: 1.15}
	\begin{tabular}{|c|c|c|c|} \hline
		$i$ & $x_{i}$ & $f(x_{i})$ & $f'(x_{i})$ \\ \hline
		$0$ & $1.1500000$ & $-0.5003089$ & $0.4482386$ \\ \hline
		$1$ & $2.2661664$ & $0.6397468$ & $0.6032816$ \\ \hline
		$2$ & $1.2057216$ & $-0.4709837$ & $0.6026869$ \\ \hline
		$3$ & $1.9871949$ & $0.4029704$ & $1.0701177$ \\ \hline
		$4$ & $1.6106285$ & $-0.0600950$ & $1.2739408$ \\ \hline
		$5$ & $1.6578011$ & $0.0004120$ & $1.2891744$ \\ \hline
		$6$ & $1.6574815$ & $0.0000000$ & $1.2891160$ \\ \hline
	\end{tabular}
\end{figure}
\begin{figure}[H]
	\centering
	\caption{Punto inicial: 1.20}
	\begin{tabular}{|c|c|c|c|} \hline
		$i$ & $x_{i}$ & $f(x_{i})$ & $f'(x_{i})$ \\ \hline
		$0$ & $1.2000000$ & $-0.4743881$ & $0.5872960$ \\ \hline
		$1$ & $2.0077496$ & $0.4246754$ & $1.0415920$ \\ \hline
		$2$ & $1.6000320$ & $-0.0735670$ & $1.2686661$ \\ \hline
		$3$ & $1.6580197$ & $0.0006939$ & $1.2892140$ \\ \hline
		$4$ & $1.6574815$ & $0.0000000$ & $1.2891160$ \\ \hline
		$5$ & $1.6574815$ & $-0.0000000$ & $1.2891160$ \\ \hline
	\end{tabular}
\end{figure}

\begin{figure}[H]
\centering
\label{grafica:1}
\begin{tikzpicture}
	\begin{axis}[ 
	title={$f(t) = 1 - \dfrac{3}{2} * e^{(-1 * (t*t - 2*t + 0.9775))} - e^{(-1 * (t*t - 8*t + 15.96))}$},
	xlabel=$t$,
	ylabel=$f(t)$,
	%ylabel={$f(t) = 1 - \dfrac{3}{2} * e^(t*t - 2*t + 0.9775) - e^(-1 * (t*t - 8*t + 15.96))$}
	xmin=-2, xmax=5,
	ymin=-1, ymax=1.5,
	minor x tick num=1,
	axis lines=center,
	axis on top=true,
	] 
	%  \addplot {x^2 - x +4}; 
	\addplot [samples=300]{1 - (1.5) * e^(-1*(x*x - 2*x + 0.9775)) - e^(-1 * (x*x - 8*x + 15.96))}; 
	% \addplot [blue, mark = *, nodes near coords=0.80,every node near coord/.style={anchor=180}] coordinates {(0.80, 0)};
	%\node[blue, ,circle,fill,inner sep=1.5pt] at (axis cs:0.80,0) {};
	\end{axis}
\end{tikzpicture}
% \caption{$f(t) = 1 - \dfrac{3}{2} * e^{(-1 * (t*t - 2*t + 0.9775))} - e^{(-1 * (t*t - 8*t + 15.96))}$}
\end{figure}

Entonces tenemos que todos los valores convergen excepto $t_{0}=0.95$, y $t_{0}=1$. Como podemos ver en \ref{grafica:1}, 
se pierde en estos valores por la pendiente de la curva ya que están
cerca de un punto critico, entonces al trazar
las rectas tangentes el método se aleja mucho.
Los cambios en $t_{0}$, a pesar de ser pequeños, la pendiente en estos puntos cambia bastante y por lo tanto también la velocidad
de convergencia.
 \newpage

% respuesta2.tex

3. Normalice hasta la BCNF la siguiente relación. Debe justificar cada paso de normalización porque, en caso contrario, no se considerará correcta la respuesta.

CONCIERTO(\underline{código\_cantante}, \underline{fecha\_presentación}, lugar, alias\_cantante, nacionalidad, nombre\_real, repertorio)

En la relación CONCIERTO se dan las siguientes dependencias funcionales entre los atributos:

(código\_cantante, fecha\_presentación) $\rightarrow$ lugar

(código\_cantante, fecha\_presentación) $\rightarrow$ repertorio

código\_cantante $\rightarrow$ alias\_cantante

código\_cantante $\rightarrow$ nacionalidad

código\_cantante $\rightarrow$ nombre\_real

(alias\_cantante, nacionalidad) $\rightarrow$ nombre\_real

lugar $\rightarrow$ repertorio


\subsection*{1NF}
Ya se encuentra en 1NF ya que todos los atributos de la relación contienen valores atómicos.


\subsection*{2NF}
Tenemos que alias\_cantante, nacionalidad y nombre\_real no dependen totalmente de la clave, por lo que procedemos a separar la relación, y tendríamos:

Concierto(\underline{código\_cantante}, \underline{fecha\_presentación}, lugar, repertorio)

Cantante(\underline{código\_cantante}, alias\_cantante, nacionalidad, nombre\_real)


\subsection*{3NF}
Tenemos que nombre\_real no depende solo de la clave primaria, por lo que:

Concierto(\underline{código\_cantante}, \underline{fecha\_presentación}, lugar, repertorio)

Cantante(\underline{código\_cantante}, alias\_cantante, nacionalidad)

Nombre(\underline{alias\_cantante}, \underline{nacionalidad}, nombre\_cantante)

 \newpage

% respuesta3.tex
 \newpage


\end{document}
