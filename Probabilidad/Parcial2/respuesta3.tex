% Pregunta 3

\section*{Pregunta 3}

Suponga un lote de 5000 fusibles eléctricos contiene 5\% de piezas defectuosas. Si se prueba una muestra de 5 fusibles, encuentre la probabilidad de hallar al menos uno defectuoso.

\vspace{0.3cm}
\textbf{R}: Tenemos que es una distribución binomial, en donde la probabilidad de éxito es $p = 0.05$, y la probabilidad del fracaso seria $q = 0.95$ y tenemos que $n = 5$ que es nuestra muestra. Y queremos saber la probabilidad de hallar al menos uno defectuoso, entonces tenemos que queremos encontrar $\prob{X \geq 1}$. Entonces tenemos:
\begin{align*}
	& \prob{1} + \prob{2} + \prob{3} + \prob{4} + \prob{5} = \prob{X \geq 1} \\
	& \prob{0} + \prob{1} + \prob{2} + \prob{3} + \prob{4} + \prob{5} = 1 \\
	& \prob{0} + \prob{X \geq 1} = 1 \\
	& \prob{X \geq 1} = 1 - \prob{0}
\end{align*}
De modo que para calcular $\prob{X \geq 1}$ solo hace falta calcular $\prob{0}$. Para calcular esto, entonces para calcular $\prob{0}$, podemos usar:
\begin{align*}
& \prob{x; n, p} = \binom{n}{x} p^x (1 - p)^{n-x}
\end{align*}
Tenemos $n$ y $p$, por lo tanto:
\begin{align*}
	& \prob{0} = \binom{5}{0} \left( 0.05^0 \right) (1 - 0.05)^{5-0} \\
	& \prob{0} = (1) (1) (0.95)^5 \\
	& \prob{0} = (1) (1) (0.95)^5 \\
	& \prob{0} = 0.95^5
\end{align*}
Con esto ya podemos calcular $\prob{X \geq 1}$, entonces:
\begin{align*}
	& \prob{X \geq 1} = 1 - 0.95^5 \\
	& \prob{X \geq 1} = 0.2262190625
\end{align*}
$\therefore$ Entonces la probabilidad de que haya al menos uno defectuoso es de 0.2262190625