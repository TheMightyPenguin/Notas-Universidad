% respuesta4.tex

Queremos saber si la regla de Simpson integra correctamente a todos los polinomios de grado menor o igual a 3, osea si integra correctamente a:
\begin{dmath*}
    p(x) = Ax^3 + Bx^2 + Cx + D
\end{dmath*}
Entonces, si calculamos la integral, tendríamos:
\begin{dmath*}
	\int_{a}^{b} (Ax^3 + Bx^2 + Cx + D) dx = 
	\frac{A}{4}(b^4 - a^4) + \frac{B}{3}(b^3 - a^3) +
	\frac{C}{2}(b^2 - a^2) + D(b - a)
\end{dmath*}
Entonces, al aplicar la regla de Simpson a $p(x)$, y desarrollando, tendríamos:
\begin{dmath*}
	\int_{a}^{b} p(x) dx \approx 
	\frac{b - a}{6} \left[ \customP{a} + 4 \left( \customP{ \left( \frac{a + b}{2} \right) } \right) + \customP{b} \right]
	= \frac{b - a}{6} \left[ \customPa{A}{^3} + \customPa{B}{^2} + \customPa{C}{} + 6D \right]
	= \frac{b - a}{6} \left[ \customPaa + \customPbb + \customPcc + 6D \right]
	= \frac{b - a}{6} \left[ \customPaaa + \customPbbb + \customPccc + 6D \right]
	= \frac{b - a}{6} \left[ \customPaaaa + \customPbbbb + \customPcccc + 6D \right]
	= \customPaaaaa + \customPbbbbb + \customPccccc + D(b-a)
	= \customPna + \customPnb + \customPnc + D(b-a)
	= \customPnaa + \customPnbb + \customPncc + D(b-a)
\end{dmath*}

$\therefore$ Se puede observar que llegamos al resultado de la integral. Y como los polinomios de grado 2, 1  y 0, son casos particulares de los polinomios de grado 3 (Cuando $A$, $B$ o $C$ son 0), queda demostrado que la regla de Simpson integra correctamente a todos los polinomios de grado menor o igual a 3 con error cero.

