% respuesta6.tex

Tenemos que

\begin{equation*}
	c(t) = A t e^{-t / 3}
\end{equation*}

y queremos saber la cantidad $A$ que debe inyectarse para lograr
la concentración máxima de $c(t) = 1 mg/mL$, entonces calculamos
la derivada de $c$ para saber en que punto alcanza el máximo.

\begin{align*}
% \text{\textbf{Por Truncamiento tenemos que:}} \\
\left(t e^{-t / 3} \right)' & = \frac{e^{-t/3}}{3} (3 - t)  \\
\end{align*}

Si igualamos la derivada a 0, obtendremos el punto máximo de la función, entonces para encontrar la solución de:

\begin{align*}
\frac{e^{-t/3}}{3} (3 - t) & = 0 \\
\end{align*}

Aqui tenemos que la solucion es $t = 3$. 
Entonces teniendo esto, y que el punto máximo de $c(t)$ debe ser 1, y remplazando $t$ por 3 tendríamos:

\begin{equation*}
	A = \frac{e}{3}
\end{equation*}

Entonces quedamos con:
\begin{equation*}
	c(t) = \frac{e}{3} t e^{-t/3}
\end{equation*}

Tenemos que se debe administrar $\frac{e}{3}$ unidades al paciente, y que toma 3 horas alcanzar la concentración máxima ($1mg/mL$).
Para saber cuando debe colocarse la segunda inyección debemos saber cuando la función desciende a $0.25mg/mL$ luego de que alcanza su máximo. Entonces tendríamos:

\begin{align*}
\frac{e^{-t/3}}{3} (3 - t) & = 0.25 \\
\frac{4e^{-t/3}}{3} (3 - t) - 1 & = 0 \\
\end{align*}

Y ahora teniendo esto, si lo derivamos y le aplicamos newton y tomamos un valor inicial de 6, tenemos que converge a  11.0779036.

Entonces tarda  $11.0779036$ horas, o lo que es lo mismo, $11$ horas con $4.20$ minutos.

