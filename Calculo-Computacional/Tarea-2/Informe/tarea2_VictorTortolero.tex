\documentclass{article}

\usepackage[spanish]{babel}
\usepackage[utf8]{inputenc}
\usepackage[T1]{fontenc}
\usepackage{graphicx}
\usepackage{wrapfig}
\usepackage{pdfpages}
\usepackage{hyperref}
\usepackage{courier}
\usepackage{longtable}
\usepackage{listings}
\usepackage{minted}
\usepackage{xcolor}
\usepackage{blindtext}
\usepackage{scrextend}
\usepackage[document]{ragged2e}
\usepackage{multicol}
\usepackage{booktabs}
\usepackage{amsmath}
 % % % % % % % %

\usemintedstyle{pastie}

\usepackage{array}
\newcolumntype{L}[1]{>{\raggedright\let\newline\\\arraybackslash\hspace{0pt}}m{#1}}
\newcolumntype{C}[1]{>{\centering\let\newline\\\arraybackslash\hspace{0pt}}m{#1}}
\newcolumntype{R}[1]{>{\raggedleft\let\newline\\\arraybackslash\hspace{0pt}}m{#1}}

%Custom commands
\newcommand\myeq{\mathrel{\overset{\makebox[0pt]{\mbox{\normalfont\tiny\sffamily 24 Ceros}}}{00\dots00}}}

\usepackage{anysize}
\marginsize{2.54cm}{2.54cm}{2.54cm}{2.54cm}

\usepackage{setspace}
\onehalfspacing
%\doublespacing

\setlength{\columnsep}{1cm}

%En caso de que LaTeX separe las palabras con - de manera incorrecta, usar
%\hyphenation{deci-sión,e-xa-men, otras palabras....}

\setlength{\columnseprule}{2pt}
\def\columnseprulecolor{\color{black}}


% Aqui comienza el documento como tal!!
\begin{document}

\flushleft
\setlength{\parindent}{20pt}

\justify
%%CUERPO PRINCIPAl%%%%%%%%%%%%%%%%%%%%%%%%%%%%%%%%%%%%%%%%%%%
\centerline{\huge Tarea 2 \textbf{Calculo Computacional}}
\centerline{Victor Tortolero CI:24.569.609}  % Pon tu nombre y Cedula!
\vspace{0.1cm}
\hrule

%--% Respuesta 1 %--------------------------------------------------%
\section*{Respuesta 1}
asd
%-------------------------------------------------------------------%


%--%FIN CUERPO PRINCIPAl %------------------------------------------%



% Codigo Fuente %%%%%%%%%%%%%%%%%%%%%%%%
%\newpage
%\begin{centering}
%	\section*{Código Fuente}
%\end{centering}

%\section*{repuesta3.c}
%\inputminted[
%frame=lines,
%framesep=2mm,
%baselinestretch=1.2,
%fontsize=\footnotesize,
%linenos
%]{c}{../Codigos/respuesta3.c}
%%%%%%%%%%%%%%%%%%%%%%%%%%%%%%%%%%%%%%%%%%%%%%%%%%%%%%

\end{document}
