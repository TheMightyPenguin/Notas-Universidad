\documentclass{article}

\usepackage[spanish]{babel}
\usepackage[utf8]{inputenc}
\usepackage[T1]{fontenc}
\usepackage{graphicx}
\usepackage{wrapfig}
\usepackage{pdfpages}
\usepackage{hyperref}
\usepackage{courier}
\usepackage{longtable}
\usepackage{listings}
\usepackage{minted}
\usepackage{xcolor}
\usepackage{blindtext}
\usepackage{scrextend}
\usepackage[document]{ragged2e}
\usepackage{multicol}
\usepackage{booktabs}
\usepackage{amsmath}
 % % % % % % % %

\usemintedstyle{pastie}

\usepackage{array}
\newcolumntype{L}[1]{>{\raggedright\let\newline\\\arraybackslash\hspace{0pt}}m{#1}}
\newcolumntype{C}[1]{>{\centering\let\newline\\\arraybackslash\hspace{0pt}}m{#1}}
\newcolumntype{R}[1]{>{\raggedleft\let\newline\\\arraybackslash\hspace{0pt}}m{#1}}

%Custom commands
\newcommand\myeq{\mathrel{\overset{\makebox[0pt]{\mbox{\normalfont\tiny\sffamily 24 Ceros}}}{00\dots00}}}

\usepackage{anysize}
\marginsize{2.54cm}{2.54cm}{2.54cm}{2.54cm}

\usepackage{setspace}
\onehalfspacing
%\doublespacing

\setlength{\columnsep}{1cm}

%En caso de que LaTeX separe las palabras con - de manera incorrecta, usar
%\hyphenation{deci-sión,e-xa-men, otras palabras....}

\setlength{\columnseprule}{2pt}
\def\columnseprulecolor{\color{black}}


% Aqui comienza el documento como tal!!
\begin{document}

\flushleft
\setlength{\parindent}{20pt}

\justify
%%CUERPO PRINCIPAl%%%%%%%%%%%%%%%%%%%%%%%%%%%%%%%%%%%%%%%%%%%
\centerline{\huge Tarea 1 \textbf{Calculo Computacional}}
\centerline{Victor Tortolero CI:24.569.609}  % Pon tu nombre y Cedula!
\vspace{0.1cm}
\hrule

%% Respuesta 1 %%%%%%%%%%%%%%%%%%	
\section*{Respuesta 1}
Tenemos que $\frac{A + 3}{13}$, como $A = 9$, tendríamos $\frac{9 + 3}{13} = \frac{12}{13}$.
Ahora procedemos a convertir a binario. \newline
\begin{multicols}{2}
	$\frac{12}{13} \times 2 = \frac{24}{13}$, $b_{0} = 1$ \newline
	$\frac{11}{13} \times 2 = \frac{22}{13}$, $b_{1} = 1$ \newline
	$\frac{9}{13} \times 2 = \frac{18}{13}$, $b_{2} = 1$ \newline
	$\frac{5}{13} \times 2 = \frac{10}{13}$, $b_{3} = 0$ \newline
	$\frac{10}{13} \times 2 = \frac{20}{13}$, $b_{4} = 1$ \newline
	$\frac{7}{13} \times 2 = \frac{14}{13}$, $b_{5} = 1$ \newline
	$\frac{1}{13} \times 2 = \frac{2}{13}$, $b_{6} = 0$ \newline
	$\frac{2}{13} \times 2 = \frac{4}{13}$, $b_{7} = 0$ \newline
	$\frac{4}{13} \times 2 = \frac{8}{13}$, $b_{8} = 0$ \newline
	$\frac{8}{13} \times 2 = \frac{16}{13}$, $b_{9} = 1$ \newline
	$\frac{3}{13} \times 2 = \frac{6}{13}$, $b_{10} = 0$ \newline
	$\frac{6}{13} \times 2 = \frac{12}{13}$, $b_{11} = 0$ \newline
\end{multicols}

Por lo tanto tenemos que:
\begin{equation*}
	0.111011000100\overline{111011000100}\textbf{1}\dots
\end{equation*}
Observemos que el numero que vendria luego del bit 24 seria un 1. Entonces a la hora de redondear se suma 1.
Por lo tanto, tenemos que $Fl(\frac{12}{13})_{Truncado} = 0.111011000100111011000100$, 
y que $Fl(\frac{12}{13})_{Redondeado} = 0.111011000100111011000101$.
\begin{align*}
	\text{\textbf{Por Truncamiento tenemos que:}} \\
	E_{A} = |x - Fl(x)_{Truncado}| & = 0, \underbrace{000\dots000}_\text{24 Ceros}111011000100\dots \\
	& = 0,\underbrace{111011000100111011000100}_\text{Esto es $\frac{12}{13}$}\dots \times 2^{-24} \\
	& = \frac{12}{13} \times 2^{-24} \approx 5,50196 \times 10^{-8} \\
	E_{R} = \frac{E_{A}}{|x|} & = \frac{\frac{12}{13} \times 2^{-24}}{\frac{12}{13}} \\
	& = 2^{-24} \approx 5,96046 \times 10^{-8}
\end{align*}

\begin{align*}
\text{\textbf{Por Redondeo tenemos que:}} \\
E_{A} = |x - Fl(x)_{Redondeado}| & = |x - (Fl(x)_{Truncado} + 1 \times 2^{-24})| \\
& = |x - Fl(x)_{Truncado} - 1 \times 2^{-24}| \\
& = |\frac{12}{13} \times 2^{-24} - 1 \times 2^{-24}|  \\
& = |\frac{12}{13} - 1| \times 2^{-24} \\
& = \frac{1}{13} \times 2^{-24} \approx 4,584 \times 10^{-9} \\
E_{R} = \frac{E_{A}}{|x|} & = \frac{\frac{1}{13} \times 2^{-24}}{\frac{12}{13}} \\
& = \frac{1}{12} \times 2^{-24} \approx 4,9670 \times 10^{-9}
\end{align*}
%%%%%%%%%%%%%%%%%%%%%%%%%%%%%%%%%%%%%%	


%% Respuesta 2 %%%%%%%%%%%%%%%%%%%%%%
\hrule
\section*{Respuesta 2}
Tenemos $245696,09_{10}$, procedemos a convertirlo a binario:

\begin{itemize}
	\item \textbf{Parte Entera}:
	\begin{multicols}{2}
		$\frac{245696}{2} = 122848$, $b_{17} = 0$ \newline
		$\frac{122848}{2} = 61424$, $b_{16} = 0$ \newline
		$\frac{61424}{2} = 30712$, $b_{15} = 0$ \newline
		$\frac{30712}{2} = 15356$, $b_{14} = 0$ \newline
		$\frac{15356}{2} = 7678$, $b_{13} = 0$ \newline
		$\frac{7678}{2} = 3839$, $b_{12} = 0$ \newline
		$\frac{3839}{2} = 1919$, $b_{11} = 1$ \newline
		$\frac{1919}{2} = 959$, $b_{10} = 1$ \newline
		$\frac{959}{2} = 479$, $b_{09} = 1$ \newline
		$\frac{479}{2} = 239$, $b_{08} = 1$ \newline
		$\frac{239}{2} = 119$, $b_{07} = 1$ \newline
		$\frac{119}{2} = 59$, $b_{06} = 1$ \newline
		$\frac{59}{2} = 29$, $b_{05} = 1$ \newline
		$\frac{29}{2} = 14$, $b_{04} = 1$ \newline
		$\frac{14}{2} = 7$, $b_{03} = 0$ \newline
		$\frac{7}{2} = 7$, $b_{02} = 1$ \newline
		$\frac{3}{2} = 1$, $b_{01} = 1$ \newline
		$\frac{1}{2} = 0$, $b_{00} = 1$ \newline
	\end{multicols}
	Por lo tanto tenemos que $245696_{10} = 111011111111000000_{2}$.
	
	\item \textbf{Parte Decimal}:
	\begin{multicols}{2}
		$\frac{9}{100} \times 2 = \frac{18}{100}$, $b_{0} = 0$ \newline
		$\frac{18}{100} \times 2 = \frac{36}{100}$, $b_{1} = 0$ \newline
		$\frac{36}{100} \times 2 = \frac{27}{100}$, $b_{2} = 0$ \newline
		$\frac{72}{100} \times 2 = \frac{144}{100}$, $b_{3} = 1$ \newline
		$\frac{44}{100} \times 2 = \frac{88}{100}$, $b_{4} = 0$ \newline
		$\frac{88}{100} \times 2 = \frac{176}{100}$, $b_{5} = 1$ \newline
		$\frac{76}{100} \times 2 = \frac{152}{100}$, $b_{6} = 1$ \newline
		$\frac{52}{100} \times 2 = \frac{104}{100}$, $b_{7} = 1$ \newline
	\end{multicols}
	Por lo tanto tenemos que $0,09_{10} = 00010111$.
\end{itemize}
{\noindent
	Entonces se tiene que $245696,09_{10} \approx \overbrace{111011111111000000,000101}^{24 bits}11_{2}$. \newline

\noindent	
	Si usamos \textbf{redondeo:} \newline
	$\boldsymbol{Fl(245696,09)_{Redondeado} = 0,111011111111000000000110 \times 2^{18}}$
	\newline

\noindent	
	Si representamos este numero de vuelta en \textbf{decimal:} \newline
	$\boldsymbol{111011111111000000,000110_{2} = 245696,09375_{10}}$. \newline
	

\noindent	
	\textbf{Error absoluto y relativo}:
	\begin{align*}
	E_{A} = |x - Fl(x)_{Redondeado}| & = |245696,09 - 245696,09375| \\
	& = -3.75 \times 10^{-3} \\
	E_{R} = \frac{E_{A}}{|x|} & = \frac{-3,75 \times 10^{-3}}{245696,09} \approx -1.526275815 \times 10^{-8} \\
	\end{align*}
}

	
%%%%%%%%%%%%%%%%%%%%%%%%%%%%%%%%%%%%%%


%% Respuesta 3 %%%%%%%%%%%%%%%%%%%%%%
\hrule
\section*{Respuesta 3}



Después de correr el programa, se obtuvieron los siguientes datos
\begin{itemize}
	\item \textbf{Para simple precisión}:
	$\epsilon = 0.0000001192092895507812500000000000$ \newline
	\begin{longtable}{|c||c|c|}
		\hline
		Iteracion & t & $\epsilon$ \\ \hline \hline
		1 & 1.500000000000000000000000 & 0.5000000000000000000000000000000000 \\  \hline
		2 & 1.250000000000000000000000 & 0.2500000000000000000000000000000000 \\  \hline
		3 & 1.125000000000000000000000 & 0.1250000000000000000000000000000000 \\  \hline
		4 & 1.062500000000000000000000 & 0.0625000000000000000000000000000000 \\  \hline
		5 & 1.031250000000000000000000 & 0.0312500000000000000000000000000000 \\  \hline
		6 & 1.015625000000000000000000 & 0.0156250000000000000000000000000000 \\  \hline
		7 & 1.007812500000000000000000 & 0.0078125000000000000000000000000000 \\  \hline
		8 & 1.003906250000000000000000 & 0.0039062500000000000000000000000000 \\  \hline
		9 & 1.001953125000000000000000 & 0.0019531250000000000000000000000000 \\  \hline
		10 & 1.000976562500000000000000 & 0.0009765625000000000000000000000000 \\  \hline
		11 & 1.000488281250000000000000 & 0.0004882812500000000000000000000000 \\  \hline
		12 & 1.000244140625000000000000 & 0.0002441406250000000000000000000000 \\  \hline
		13 & 1.000122070312500000000000 & 0.0001220703125000000000000000000000 \\  \hline
		14 & 1.000061035156250000000000 & 0.0000610351562500000000000000000000 \\  \hline
		15 & 1.000030517578125000000000 & 0.0000305175781250000000000000000000 \\  \hline
		16 & 1.000015258789062500000000 & 0.0000152587890625000000000000000000 \\  \hline
		17 & 1.000007629394531200000000 & 0.0000076293945312500000000000000000 \\  \hline
		18 & 1.000003814697265600000000 & 0.0000038146972656250000000000000000 \\  \hline
		19 & 1.000001907348632800000000 & 0.0000019073486328125000000000000000 \\  \hline
		20 & 1.000000953674316400000000 & 0.0000009536743164062500000000000000 \\  \hline
		21 & 1.000000476837158200000000 & 0.0000004768371582031250000000000000 \\  \hline
		22 & 1.000000238418579100000000 & 0.0000002384185791015625000000000000 \\  \hline
		23 & 1.000000119209289600000000 & 0.0000001192092895507812500000000000 \\  \hline
		24 & 1.000000000000000000000000 & 0.0000000596046447753906250000000000 \\  \hline
	\end{longtable}
	
	\vspace{0.2cm}
	\item \textbf{Para doble precisión}:
	$\epsilon =  0.0000000000000002220446049250313100$ \newline
	\begin{longtable}{|c||c|c|}
		\hline
		Iteracion & t & $\epsilon$ \\ \hline \hline
		1 & 1.500000000000000000000000 & 0.5000000000000000000000000000000000 \\ \hline 
		2 & 1.250000000000000000000000 & 0.2500000000000000000000000000000000 \\ \hline 
		3 & 1.125000000000000000000000 & 0.1250000000000000000000000000000000 \\ \hline 
		4 & 1.062500000000000000000000 & 0.0625000000000000000000000000000000 \\ \hline 
		5 & 1.031250000000000000000000 & 0.0312500000000000000000000000000000 \\ \hline 
		6 & 1.015625000000000000000000 & 0.0156250000000000000000000000000000 \\ \hline 
		7 & 1.007812500000000000000000 & 0.0078125000000000000000000000000000 \\ \hline 
		8 & 1.003906250000000000000000 & 0.0039062500000000000000000000000000 \\ \hline 
		9 & 1.001953125000000000000000 & 0.0019531250000000000000000000000000 \\ \hline 
		10 & 1.000976562500000000000000 & 0.0009765625000000000000000000000000 \\ \hline 
		11 & 1.000488281250000000000000 & 0.0004882812500000000000000000000000 \\ \hline 
		12 & 1.000244140625000000000000 & 0.0002441406250000000000000000000000 \\ \hline 
		13 & 1.000122070312500000000000 & 0.0001220703125000000000000000000000 \\ \hline 
		14 & 1.000061035156250000000000 & 0.0000610351562500000000000000000000 \\ \hline 
		15 & 1.000030517578125000000000 & 0.0000305175781250000000000000000000 \\ \hline 
		16 & 1.000015258789062500000000 & 0.0000152587890625000000000000000000 \\ \hline 
		17 & 1.000007629394531200000000 & 0.0000076293945312500000000000000000 \\ \hline 
		18 & 1.000003814697265600000000 & 0.0000038146972656250000000000000000 \\ \hline 
		19 & 1.000001907348632800000000 & 0.0000019073486328125000000000000000 \\ \hline 
		20 & 1.000000953674316400000000 & 0.0000009536743164062500000000000000 \\ \hline 
		21 & 1.000000476837158200000000 & 0.0000004768371582031250000000000000 \\ \hline 
		22 & 1.000000238418579100000000 & 0.0000002384185791015625000000000000 \\ \hline 
		23 & 1.000000119209289600000000 & 0.0000001192092895507812500000000000 \\ \hline 
		24 & 1.000000059604644800000000 & 0.0000000596046447753906250000000000 \\ \hline 
		25 & 1.000000029802322400000000 & 0.0000000298023223876953130000000000 \\ \hline 
		26 & 1.000000014901161200000000 & 0.0000000149011611938476560000000000 \\ \hline 
		27 & 1.000000007450580600000000 & 0.0000000074505805969238281000000000 \\ \hline 
		28 & 1.000000003725290300000000 & 0.0000000037252902984619141000000000 \\ \hline 
		29 & 1.000000001862645100000000 & 0.0000000018626451492309570000000000 \\ \hline 
		30 & 1.000000000931322600000000 & 0.0000000009313225746154785200000000 \\ \hline 
		31 & 1.000000000465661300000000 & 0.0000000004656612873077392600000000 \\ \hline 
		32 & 1.000000000232830600000000 & 0.0000000002328306436538696300000000 \\ \hline 
		33 & 1.000000000116415300000000 & 0.0000000001164153218269348100000000 \\ \hline 
		34 & 1.000000000058207700000000 & 0.0000000000582076609134674070000000 \\ \hline 
		35 & 1.000000000029103800000000 & 0.0000000000291038304567337040000000 \\ \hline 
		36 & 1.000000000014551900000000 & 0.0000000000145519152283668520000000 \\ \hline 
		37 & 1.000000000007276000000000 & 0.0000000000072759576141834259000000 \\ \hline 
		38 & 1.000000000003638000000000 & 0.0000000000036379788070917130000000 \\ \hline 
		39 & 1.000000000001819000000000 & 0.0000000000018189894035458565000000 \\ \hline 
		40 & 1.000000000000909500000000 & 0.0000000000009094947017729282400000 \\ \hline 
		41 & 1.000000000000454700000000 & 0.0000000000004547473508864641200000 \\ \hline 
		42 & 1.000000000000227400000000 & 0.0000000000002273736754432320600000 \\ \hline 
		43 & 1.000000000000113700000000 & 0.0000000000001136868377216160300000 \\ \hline 
		44 & 1.000000000000056800000000 & 0.0000000000000568434188608080150000 \\ \hline 
		45 & 1.000000000000028400000000 & 0.0000000000000284217094304040070000 \\ \hline 
		46 & 1.000000000000014200000000 & 0.0000000000000142108547152020040000 \\ \hline 
		47 & 1.000000000000007100000000 & 0.0000000000000071054273576010019000 \\ \hline 
		48 & 1.000000000000003600000000 & 0.0000000000000035527136788005009000 \\ \hline 
		49 & 1.000000000000001800000000 & 0.0000000000000017763568394002505000 \\ \hline 
		50 & 1.000000000000000900000000 & 0.0000000000000008881784197001252300 \\ \hline 
		51 & 1.000000000000000400000000 & 0.0000000000000004440892098500626200 \\ \hline 
		52 & 1.000000000000000200000000 & 0.0000000000000002220446049250313100 \\ \hline 
		53 & 1.000000000000000000000000 & 0.0000000000000001110223024625156500 \\ \hline 
	\end{longtable}
	
	El valor de $\epsilon$ es distinto de $10^{-308}$, porque como estamos continuamente sumando
	1 con $\epsilon$, y $\epsilon$ se vuelve mas pequeño con cada iteración, y su magnitud es muy pequeña
	comparada con la de 1 y la suma de $1 + \epsilon$ deja de ser significativa.
	
	Para precisión simple $\delta = 0.00097656250$ y para precisión doble $\delta = 0.00000000000181898940354585650.$
	
	Los valores de $\epsilon$ y $\delta$ son distintos ya que la magnitud de 10000 es mucho mayor a la de 1 y por lo tanto
	al sumarle números pequeños se llega de manera rápida a uno que no afecte la suma.
\end{itemize}
%%%%%%%%%%%%%%%%%%%%%%%%%%%%%%%%%%%%%%


%% Respuesta 4 %%%%%%%%%%%%%%%%%%%%%%
\hrule
\section*{Respuesta 4}
\begin{itemize}
	\item Ascendente precisión simple:
	\item Ascendente precisión doble:
	\item Descendente precisión simple:
	\item Descendente precisión doble:
	\item Mayor a menor precisión simple:
	\item Mayor a menor precisión doble:
	\item Menor a mayor precisión simple:
	\item Menor a mayor precisión doble:
\end{itemize}
%%%%%%%%%%%%%%%%%%%%%%%%%%%%%%%%%%%%%%


%% Respuesta 5 %%%%%%%%%%%%%%%%%%%%%%
\hrule
\section*{Respuesta 5}
El mayor valor que llego a tomar la sumatoria fue \textbf{15.4036827008740234}.
Fueron sumados \textbf{2097152 términos} antes de que la computadora dejara de "sumar".

La computadora no llega infinito al realizar la sumatoria debido
a la precisión decimal, llega a un punto en que la computadora al sumar dos números, las magnitudes entre ellos
son muy distintas y por lo tanto se queda con el numero mas grande y es como si no se le sumara nada.
\vspace{0.3cm}
%%%%%%%%%%%%%%%%%%%%%%%%%%%%%%%%%%%%%%


%% Respuesta 6 %%%%%%%%%%%%%%%%%%%%%%
\hrule
\section*{Respuesta 6}
Para $x = 10$, con precisión simple tenemos que $\boldsymbol{e^{10} = 22026.4667968750}$,
este resultado se obtuvo al sumar los términos desde un $n = 0$, y hasta que la suma dejara de "sumar",
usando al final \textbf{32 iteraciones}.

Y para precisión doble tenemos $\boldsymbol{e^{10} = 22026.465760913433769019320607185363769531250}$.
que se obtuvo al cambiar el orden en que se suma y se empezó desde $n = 32$ hasta $n = 0$. 
En este caso usamos \textbf{32 iteraciones}.

Si se empezara desde un $n$ muy grande y hasta $n = 0$, se tendría un resultado mas preciso.
%%%%%%%%%%%%%%%%%%%%%%%%%%%%%%%%%%%%%%%%

%% \includepdf[scale=0.8,pages=1,pagecommand=\subsection*{Enunciado Original}]{../Primera_tarea_Semestre_I_2016}

%%FIN CUERPO PRINCIPAl%%%%%%%%%%%%%%%%%%



% Codigo Fuente %%%%%%%%%%%%%%%%%%%%%%%%
\newpage
\begin{centering}
	\section*{Código Fuente}
\end{centering}

\section*{repuesta3.c}
\inputminted[
frame=lines,
framesep=2mm,
baselinestretch=1.2,
fontsize=\footnotesize,
linenos
]{c}{../Codigos/respuesta3.c}

\newpage
\section*{repuesta4.c}
\inputminted[
frame=lines,
framesep=2mm,
baselinestretch=1.2,
fontsize=\footnotesize,
linenos
]{c}{../Codigos/respuesta4.c}

\newpage
\section*{repuesta5.c}
\inputminted[
frame=lines,
framesep=2mm,
baselinestretch=1.2,
fontsize=\footnotesize,
linenos
]{c}{../Codigos/respuesta5.c}

\newpage
\section*{repuesta6.c}
\inputminted[
frame=lines,
framesep=2mm,
baselinestretch=1.2,
fontsize=\footnotesize,
linenos
]{c}{../Codigos/respuesta6.c}
%%%%%%%%%%%%%%%%%%%%%%%%%%%%%%%%%%%%%%%%%%%%%%%%%%%%%%

\end{document}
