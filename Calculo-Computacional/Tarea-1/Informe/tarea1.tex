\documentclass{article}

\usepackage[spanish]{babel}
\usepackage[utf8]{inputenc}
\usepackage[T1]{fontenc}
\usepackage{graphicx}
\usepackage{wrapfig}
\usepackage{hyperref}
\usepackage{courier}
\usepackage{listings}
\usepackage{minted}
\usepackage{xcolor}
\usepackage{blindtext}
\usepackage{scrextend}
\usepackage[document]{ragged2e}
\usepackage{multicol}
\usepackage{booktabs}
\usepackage{amsmath}
 % % % % % % % %

\usemintedstyle{pastie}

\usepackage{array}
\newcolumntype{L}[1]{>{\raggedright\let\newline\\\arraybackslash\hspace{0pt}}m{#1}}
\newcolumntype{C}[1]{>{\centering\let\newline\\\arraybackslash\hspace{0pt}}m{#1}}
\newcolumntype{R}[1]{>{\raggedleft\let\newline\\\arraybackslash\hspace{0pt}}m{#1}}

%Custom commands
\newcommand\myeq{\mathrel{\overset{\makebox[0pt]{\mbox{\normalfont\tiny\sffamily 24 Ceros}}}{00\dots00}}}

\usepackage{anysize}
\marginsize{2.54cm}{2.54cm}{2.54cm}{2.54cm}

\usepackage{setspace}
\onehalfspacing
%\doublespacing

\setlength{\columnsep}{1cm}

%En caso de que LaTeX separe las palabras con - de manera incorrecta, usar
%\hyphenation{deci-sión,e-xa-men, otras palabras....}

\setlength{\columnseprule}{2pt}
\def\columnseprulecolor{\color{black}}


% Aqui comienza el documento como tal!!
\begin{document}

\flushleft
\setlength{\parindent}{20pt}

\justify
%%CUERPO PRINCIPAl%%%%%%%%%%%%%%%%%%%%%%%%%%%%%%%%%%%%%%%%%%%
\centerline{\huge Tarea 1 \textbf{Calculo Computacional}}
\centerline{Victor Tortolero CI:24.569.609}  % Pon tu nombre y Cedula!
\hrule

%% Respuesta 1 %%%%%%%%%%%%%%%%%%	
\section*{Respuesta 1}
Tenemos que $\frac{A + 3}{13}$, como $A = 9$, tendríamos $\frac{9 + 3}{13} = \frac{12}{13}$.
Ahora procedemos a convertir a binario. \newline
\begin{multicols}{2}
	$\frac{12}{13} \times 2 = \frac{24}{13}$, $b_{0} = 1$ \newline
	$\frac{11}{13} \times 2 = \frac{22}{13}$, $b_{1} = 1$ \newline
	$\frac{9}{13} \times 2 = \frac{18}{13}$, $b_{2} = 1$ \newline
	$\frac{5}{13} \times 2 = \frac{10}{13}$, $b_{3} = 0$ \newline
	$\frac{10}{13} \times 2 = \frac{20}{13}$, $b_{4} = 1$ \newline
	$\frac{7}{13} \times 2 = \frac{14}{13}$, $b_{5} = 1$ \newline
	$\frac{1}{13} \times 2 = \frac{2}{13}$, $b_{6} = 0$ \newline
	$\frac{2}{13} \times 2 = \frac{4}{13}$, $b_{7} = 0$ \newline
	$\frac{4}{13} \times 2 = \frac{8}{13}$, $b_{8} = 0$ \newline
	$\frac{8}{13} \times 2 = \frac{16}{13}$, $b_{9} = 1$ \newline
	$\frac{3}{13} \times 2 = \frac{6}{13}$, $b_{10} = 0$ \newline
	$\frac{6}{13} \times 2 = \frac{12}{13}$, $b_{11} = 0$ \newline
\end{multicols}

Por lo tanto tenemos que:
\begin{equation*}
	0.111011000100\overline{111011000100}\textbf{1}\dots
\end{equation*}
Observemos que el numero que vendria luego del bit 24 seria un 1. Entonces a la hora de redondear se suma 1.
Por lo tanto, tenemos que $Fl(\frac{12}{13})_{Truncado} = 0.111011000100111011000100$, 
y que $Fl(\frac{12}{13})_{Redondeado} = 0.111011000100111011000101$.
\begin{align*} % aqui empieza la locura xD
	\text{\textbf{Por Truncamiento tenemos que:}} \\
	E_{A} = |x - Fl(x)_{Truncado}| & = 0, \underbrace{000\dots000}_\text{24 Ceros}111011000100\dots \\
	& = 0,\underbrace{111011000100111011000100}_\text{Esto es $\frac{12}{13}$}\dots \times 2^{-24} \\
	& = \frac{12}{13} \times 2^{-24} \approx 5,50196 \times 10^{-8} \\
	E_{R} = \frac{E_{A}}{|X|} & = \frac{\frac{12}{13} \times 2^{-24}}{\frac{12}{13}} \\
	& = 2^{-24} \approx 5,96046 \times 10^{-8}
\end{align*}

\begin{align*}
\text{\textbf{Por Redondeo tenemos que:}} \\
E_{A} = |x - Fl(x)_{Redondeado}| & = |x - (Fl(x)_{Truncado} + 1 \times 2^{-24})| \\
& = |x - Fl(x)_{Truncado} - 1 \times 2^{-24}| \\
& = |\frac{12}{13} \times 2^{-24} - 1 \times 2^{-24}|  \\
& = |\frac{12}{13} - 1| \times 2^{-24} \\
& = \frac{1}{13} \times 2^{-24} \approx 4,584 \times 10^{-9} \\
E_{R} = \frac{E_{A}}{|X|} & = \frac{\frac{1}{13} \times 2^{-24}}{\frac{12}{13}} \\
& = \frac{1}{12} \times 2^{-24} \approx 4,9670 \times 10^{-9}
\end{align*}
%%%%%%%%%%%%%%%%%%%%%%%%%%%%%%%%%%%%%%	


%% Respuesta 2 %%%%%%%%%%%%%%%%%%%%%%
\hrule
\section*{Respuesta 2}

%%%%%%%%%%%%%%%%%%%%%%%%%%%%%%%%%%%%%%


%% Respuesta 3 %%%%%%%%%%%%%%%%%%%%%%
\hrule
\section*{Respuesta 3}

%%%%%%%%%%%%%%%%%%%%%%%%%%%%%%%%%%%%%%


%% Respuesta 4 %%%%%%%%%%%%%%%%%%%%%%
\hrule
\section*{Respuesta 4}

%%%%%%%%%%%%%%%%%%%%%%%%%%%%%%%%%%%%%%


%% Respuesta 5 %%%%%%%%%%%%%%%%%%%%%%
\hrule
\section*{Respuesta 5}
El mayor valor que llego a tomar la sumatoria fue \textbf{15.4036827008740234}.
Fueron sumados \textbf{2097152 terminos}. 
La computadora no llega infinito al realizar la sumatoria debido
a la precisión decimal, llega a un punto en que la computadora al sumar dos números, las magnitudes entre ellos
son muy distintas y por lo tanto se queda con el numero mas grande y es como si no se le sumara nada.
%%%%%%%%%%%%%%%%%%%%%%%%%%%%%%%%%%%%%%


%% Respuesta 6 %%%%%%%%%%%%%%%%%%%%%%
\hrule
\section*{Respuesta 6}
\begin{itemize}
	\item \textbf{Con precisión simple}:
	La mejor aproximación que se obtuvo para $x = 10$ fue $\boldsymbol{e^{10} = 22026.4667968750}$.
	Este resultado se obtuvo al sumar los términos desde un $n = 0$, y hasta que la suma dejara de "sumar" por
	las limitaciones de la precisión, y se llego a las \textbf{32 iteraciones}.
	
	Al cambiar el orden en que se suman los resultados, empezó desde $n = 32$ y se fue hasta $n = 0$, se obtuvo
	$\boldsymbol{e^{10} = 22026.464843750}$.
	
	\item \textbf{Con precisión doble}:
	La mejor aproximación que se obtuvo para $x = 10$ fue $\boldsymbol{e^{10} = 22026.4667968750}$.
	Este resultado se obtuvo al sumar los términos desde un $n = 0$, y hasta que la suma dejara de "sumar" por
	las limitaciones de la precisión, y se llego a las \textbf{32 iteraciones}.
	
	Al cambiar el orden en que se suman los resultados, empezó desde $n = 32$ y se fue hasta $n = 0$, se obtuvo
	$\boldsymbol{e^{10} = 22026.464843750}$. 
\end{itemize}
%%%%%%%%%%%%%%%%%%%%%%%%%%%%%%%%%%%%%%%%

%%FIN CUERPO PRINCIPAl%%%%%%%%%%%%%%%%%%



% Codigo Fuente %%%%%%%%%%%%%%%%%%%%%%%%
\newpage
\begin{centering}
	\section*{Código Fuente}
\end{centering}

\section*{repuesta3.c}
\inputminted[
frame=lines,
framesep=2mm,
baselinestretch=1.2,
fontsize=\footnotesize,
linenos
]{c}{../Codigos/respuesta3.c}

\newpage
\section*{repuesta4.c}
\inputminted[
frame=lines,
framesep=2mm,
baselinestretch=1.2,
fontsize=\footnotesize,
linenos
]{c}{../Codigos/respuesta4.c}

\newpage
\section*{repuesta5.c}
\inputminted[
frame=lines,
framesep=2mm,
baselinestretch=1.2,
fontsize=\footnotesize,
linenos
]{c}{../Codigos/respuesta5.c}

\newpage
\section*{repuesta6.c}
\inputminted[
frame=lines,
framesep=2mm,
baselinestretch=1.2,
fontsize=\footnotesize,
linenos
]{c}{../Codigos/respuesta6.c}
%%%%%%%%%%%%%%%%%%%%%%%%%%%%%%%%%%%%%%%%%%%%%%%%%%%%%%

\end{document}
