% respuesta5.tex

Normalmente el método de falsa posición calculamos el próximo
punto ($x_{i+1}$) usando la siguiente formula:
\begin{align*}
	x_{i+1} = x_{i-1} - \frac{f(x_{i-1})}{ \left( \frac{f(x_{i}) - f(x_{i-1})}{x_{i} - x_{i-1}}  \right) }
\end{align*}
Si cambiamos la formula usada en el método por la siguiente:
\begin{align*}
	x_{i+1} = \frac{ \function{f}{x_{i}} x_{i-1} - \frac{1}{2} \function{f}{x_{i-1}} x_{i} } 
	{ \function{f}{x_{i}} - \frac{1}{2} \function{f}{x_{i-1}} }
\end{align*}
Obtenemos los siguientes resultados:

\begin{center}
	\captionof{figure}{Punto inicial: -1.00}
	\begin{longtable}{|c|c|c|c|c|c|c|} \hline
		$i$ & $x_{i-1}$ & $x_{i}$ & $x_{i+1}$ & $f(x_{i-1})$ & $f(x_{i})$ & $f(x_{i+1})$ \\ \hline
		$0$ & $-1.0000000$ & $1.0000000$ & $0.7309028$ & $-11.6268222$ & $0.9037901$ & $0.7992977$ \\ \hline
		$1$ & $-1.0000000$ & $0.7309028$ & $0.5216834$ & $-11.6268222$ & $0.7992977$ & $0.6828335$ \\ \hline
		$2$ & $-1.0000000$ & $0.5216834$ & $0.3617362$ & $-11.6268222$ & $0.6828335$ & $0.4976389$ \\ \hline
		$3$ & $-1.0000000$ & $0.3617362$ & $0.2543606$ & $-11.6268222$ & $0.4976389$ & $0.2992127$ \\ \hline
		$4$ & $-1.0000000$ & $0.2543606$ & $0.1929597$ & $-11.6268222$ & $0.2992127$ & $0.1521799$ \\ \hline
		$5$ & $-1.0000000$ & $0.1929597$ & $0.1625277$ & $-11.6268222$ & $0.1521799$ & $0.0691411$ \\ \hline
		$6$ & $-1.0000000$ & $0.1625277$ & $0.1488639$ & $-11.6268222$ & $0.0691411$ & $0.0295381$ \\ \hline
		$7$ & $-1.0000000$ & $0.1488639$ & $0.1430560$ & $-11.6268222$ & $0.0295381$ & $0.0122581$ \\ \hline
		$8$ & $-1.0000000$ & $0.1430560$ & $0.1406508$ & $-11.6268222$ & $0.0122581$ & $0.0050233$ \\ \hline
		$9$ & $-1.0000000$ & $0.1406508$ & $0.1396660$ & $-11.6268222$ & $0.0050233$ & $0.0020477$ \\ \hline
		$10$ & $-1.0000000$ & $0.1396660$ & $0.1392647$ & $-11.6268222$ & $0.0020477$ & $0.0008329$ \\ \hline
		$11$ & $-1.0000000$ & $0.1392647$ & $0.1391015$ & $-11.6268222$ & $0.0008329$ & $0.0003385$ \\ \hline
		$12$ & $-1.0000000$ & $0.1391015$ & $0.1390352$ & $-11.6268222$ & $0.0003385$ & $0.0001375$ \\ \hline
		$13$ & $-1.0000000$ & $0.1390352$ & $0.1390083$ & $-11.6268222$ & $0.0001375$ & $0.0000559$ \\ \hline
		$14$ & $-1.0000000$ & $0.1390083$ & $0.1389973$ & $-11.6268222$ & $0.0000559$ & $0.0000227$ \\ \hline
		$15$ & $-1.0000000$ & $0.1389973$ & $0.1389929$ & $-11.6268222$ & $0.0000227$ & $0.0000092$ \\ \hline
		$16$ & $-1.0000000$ & $0.1389929$ & $0.1389911$ & $-11.6268222$ & $0.0000092$ & $0.0000037$ \\ \hline
		$17$ & $-1.0000000$ & $0.1389911$ & $0.1389904$ & $-11.6268222$ & $0.0000037$ & $0.0000015$ \\ \hline
		$18$ & $-1.0000000$ & $0.1389904$ & $0.1389901$ & $-11.6268222$ & $0.0000015$ & $0.0000006$ \\ \hline
		$19$ & $-1.0000000$ & $0.1389901$ & $0.1389899$ & $-11.6268222$ & $0.0000006$ & $0.0000003$ \\ \hline
		$20$ & $-1.0000000$ & $0.1389899$ & $0.1389899$ & $-11.6268222$ & $0.0000003$ & $0.0000001$ \\ \hline
		$21$ & $-1.0000000$ & $0.1389899$ & $0.1389899$ & $-11.6268222$ & $0.0000001$ & $0.0000000$ \\ \hline
		$22$ & $-1.0000000$ & $0.1389899$ & $0.1389899$ & $-11.6268222$ & $0.0000000$ & $0.0000000$ \\ \hline
		$23$ & $-1.0000000$ & $0.1389899$ & $0.1389899$ & $-11.6268222$ & $0.0000000$ & $0.0000000$ \\ \hline
	\end{longtable}
\end{center}

\vspace{1cm}

Esta nueva formula es equivalente a:
\begin{equation*}
	x_{i+1} = x_{i-1} - \frac{\function{f}{x_{i-1}}}
	{\left( \frac{2 \function{f}{x_{i}} - \function{f}{x_{i-1}}}
	{x_{i} - x_{i-1}} \right)}
\end{equation*}

Tenemos que esta formula es muy parecida a la formula original.
Lo que cambia es el $2 \function{f}{x_{i}}$. Esto hace que la recta
se incline mas hacia arriba, y por lo tanto este mas cerca del resultado, por lo que logramos converger de manera mas rápida en este caso usando esta formula.