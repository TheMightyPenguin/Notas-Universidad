% respuesta2.tex

3. Normalice hasta la BCNF la siguiente relación. Debe justificar cada paso de normalización porque, en caso contrario, no se considerará correcta la respuesta.

CONCIERTO(\underline{código\_cantante}, \underline{fecha\_presentación}, lugar, alias\_cantante, nacionalidad, nombre\_real, repertorio)

En la relación CONCIERTO se dan las siguientes dependencias funcionales entre los atributos:

(código\_cantante, fecha\_presentación) $\rightarrow$ lugar

(código\_cantante, fecha\_presentación) $\rightarrow$ repertorio

código\_cantante $\rightarrow$ alias\_cantante

código\_cantante $\rightarrow$ nacionalidad

código\_cantante $\rightarrow$ nombre\_real

(alias\_cantante, nacionalidad) $\rightarrow$ nombre\_real

lugar $\rightarrow$ repertorio


\subsection*{1NF}
Ya se encuentra en 1NF ya que todos los atributos de la relación contienen valores atómicos.

\begin{figure}[H]
	\centering
	\begin{overpic}[scale=0.65]{img/2-1nf.png}
		\put (60, 48) {\textbf{CONCIERTO}}
	\end{overpic}
\end{figure}

\subsection*{2NF}
Tenemos que alias\_cantante, nacionalidad y nombre\_real no dependen totalmente de la clave, por lo que procedemos a separar la relación, y tendríamos:

Concierto(\underline{código\_cantante}, \underline{fecha\_presentación}, lugar, repertorio)

Cantante(\underline{código\_cantante}, alias\_cantante, nacionalidad, nombre\_real)

\begin{figure}[H]
	\centering
	\begin{overpic}[scale=0.62,]{img/2-2nf_Concierto.png}
		\put (35, 38) {\textbf{Concierto}}
	\end{overpic}
	\hspace{0.4cm} \vrule \hspace{0.4cm}
	\begin{overpic}[scale=0.62,]{img/2-2nf_Cantante.png}
		\put (20, 30) {\textbf{Cantante}}
	\end{overpic}
\end{figure}

\subsection*{3NF}
Tenemos que nombre\_real no depende solo de la clave primaria y tiene una dependencia funcional transitiva, y esto viola la 3NF, por lo que crearemos una nueva relación para resolver esto:

Concierto(\underline{código\_cantante}, \underline{fecha\_presentación}, lugar)

Cantante(\underline{código\_cantante}, alias\_cantante, nacionalidad)

Nombre(\underline{alias\_cantante}, \underline{nacionalidad}, nombre\_real)

Cancion(\underline{lugar}, repertorio)

\vspace{0.35cm}
\begin{figure}[H]
	\centering
	\begin{overpic}[scale=0.63,]{img/2-3nf_Concierto.png}
		\put (33, 40) {\textbf{Concierto}}
	\end{overpic}
	\hspace{0.4cm} \vrule \hspace{0.4cm}
	\begin{overpic}[scale=0.63,]{img/2-3nf_Cantante.png}
		\put (20, 30) {\textbf{Cantante}}
	\end{overpic}
\end{figure}

\hrule \vspace{0.2cm}

\begin{figure}[H]
	\centering	
	\begin{overpic}[scale=0.63,]{img/2-3nf_Nombre.png}
		\put (20, 35) {\textbf{Nombre}}
	\end{overpic}
	\hspace{0.4cm} \vrule \hspace{0.4cm}
	\begin{overpic}[scale=0.63,]{img/2-3nf_Cancion.png}
		\put (20, 35) {\textbf{Cancion}}
	\end{overpic}
\end{figure}


\subsection*{BCNF}
Como ningún atributo no clave es determinante, tenemos que nuestras relaciones se encuentran en BCNF.
