% respuesta2.tex

3. Normalice hasta la BCNF la siguiente relación. Debe justificar cada paso de normalización porque, en caso contrario, no se considerará correcta la respuesta.

CONCIERTO(\underline{código\_cantante}, \underline{fecha\_presentación}, lugar, alias\_cantante, nacionalidad, nombre\_real, repertorio)

En la relación CONCIERTO se dan las siguientes dependencias funcionales entre los atributos:

(código\_cantante, fecha\_presentación) $\rightarrow$ lugar

(código\_cantante, fecha\_presentación) $\rightarrow$ repertorio

código\_cantante $\rightarrow$ alias\_cantante

código\_cantante $\rightarrow$ nacionalidad

código\_cantante $\rightarrow$ nombre\_real

(alias\_cantante, nacionalidad) $\rightarrow$ nombre\_real

lugar $\rightarrow$ repertorio


\subsection*{1NF}
Ya se encuentra en 1NF ya que todos los atributos de la relación contienen valores atómicos.


\subsection*{2NF}
Tenemos que alias\_cantante, nacionalidad y nombre\_real no dependen totalmente de la clave, por lo que procedemos a separar la relación, y tendríamos:

Concierto(\underline{código\_cantante}, \underline{fecha\_presentación}, lugar, repertorio)

Cantante(\underline{código\_cantante}, alias\_cantante, nacionalidad, nombre\_real)


\subsection*{3NF}
Tenemos que nombre\_real no depende solo de la clave primaria, por lo que:

Concierto(\underline{código\_cantante}, \underline{fecha\_presentación}, lugar, repertorio)

Cantante(\underline{código\_cantante}, alias\_cantante, nacionalidad)

Nombre(\underline{alias\_cantante}, \underline{nacionalidad}, nombre\_cantante)

