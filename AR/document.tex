%%%%%%%%%%%%%%%%%%%%%%%%%%%%%%%%%%%%%%%%%%%%%%%%%%%%%%%%%%%%%%%%%%%%
%% I, the copyright holder of this work, release this work into the
%% public domain. This applies worldwide. In some countries this may
%% not be legally possible; if so: I grant anyone the right to use
%% this work for any purpose, without any conditions, unless such
%% conditions are required by law.
%%%%%%%%%%%%%%%%%%%%%%%%%%%%%%%%%%%%%%%%%%%%%%%%%%%%%%%%%%%%%%%%%%%%%%%%%%%%%%%%%%
%% I, the copyright holder of this work, release this work into the
%% public domain. This applies worldwide. In some countries this may
%% not be legally possible; if so: I grant anyone the right to use
%% this work for any purpose, without any conditions, unless such
%% conditions are required by law.
%%%%%%%%%%%%%%%%%%%%%%%%%%%%%%%%%%%%%%%%%%%%%%%%%%%%%%%%%%%%%%%%%%%%

\documentclass{beamer}
% \usetheme[faculty=law]{fibeamer}
%--% Paquetes %----------------------------------%
\usepackage[utf8]{inputenc}
\usepackage[T1]{fontenc}
\usepackage{graphicx}
\usepackage{hyperref}
\usepackage{courier}
\usepackage{xcolor}
\usepackage{blindtext}
\usepackage{multicol}
\usepackage{wrapfig,lipsum}
%------------------------------------------------%

\usepackage[
main=spanish, %% By using `czech` or `slovak` as the main locale
%% instead of `english`, you can typeset the
%% presentation in either Czech or Slovak,
%% respectively.
czech, slovak %% The additional keys allow foreign texts to be
]{babel}        %% typeset as follows:
%%
%%   \begin{otherlanguage}{czech}   ... \end{otherlanguage}
%%   \begin{otherlanguage}{slovak}  ... \end{otherlanguage}
%%
%% These macros specify information about the presentation

\title{Algoritmos de Planificación del Procesador}
\subtitle{Presentation Subtitle} %% title page.
\author{Victor Tortolero, 24.569.609}
\institute{
	Sistemas Operativos, FACYT
}
\date{\today}

%% These additional packages are used within the document:
\usepackage{ragged2e}  % `\justifying` text
\usepackage{booktabs}  % Tables
\usepackage{tabularx}
\usepackage{tikz}      % Diagrams
\usetikzlibrary{calc, shapes, backgrounds}
\usepackage{amsmath, amssymb}
\usepackage{url}       % `\url`s
\usepackage{listings}  % Code listings
\frenchspacing

\begin{document}
	\frame{\maketitle}
	
	\AtBeginSection[]{% Print an outline at the beginning of sections
		\begin{frame}<beamer>
			\frametitle{Outline for Section \thesection}
			\tableofcontents[currentsection]
		\end{frame}}
		
		% \setbeamertemplate{caption}{\raggedright\insertcaption\par}
		
		%--% FCSFS %---------------------------------------------------------------------%
		\begin{frame}
			\frametitle{\algTitle ¿Que es Realidad Aumentada?}
			Se atiende a los procesos por orden de llegada, usando una cola. Es no apropiativo.
			
			\begin{itemize}
				\item \textbf{Ventajas}
				\begin{itemize}
					\item Es fácil de entender e implementar.
				\end{itemize}
				\vspace{0.5cm}
				
				\item \textbf{Desventajas}
				\begin{itemize}
					\item Aunque normalmente es justo en como dedica tiempo de CPU a los procesos,
					los procesos largos hacen esperar a los cortos.
					\item El tiempo de espera es alto por lo que carece de rendimiento.
				\end{itemize}
			\end{itemize}
		\end{frame}
		%--------------------------------------------------------------------------------%
	\end{document}
\end{document}